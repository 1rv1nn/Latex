\documentclass[a4paper,10pt]{article} 
\usepackage[top=2cm,bottom=2cm,left=2cm,rigth=2cm,heightrounded]{geometry}
\usepackage[utf8]{inputenc}
\usepackage{graphicx}
\usepackage{multirow} 
\usepackage[spanish]{babel}
\usepackage[usenames]{color}
\usepackage{dsfont}
\usepackage{amssymb}
\usepackage{amsmath}
\usepackage{bbding}  
\usepackage[dvipsnames]{xcolor}
\usepackage{csquotes}
\usepackage[export]{adjustbox}
\usepackage[all]{nowidow} 
\usepackage{csquotes} 
\everymath{\displaystyle}
\usepackage{setspace}
\usepackage[yyyymmdd]{datetime} 
\renewcommand{\dateseparator}{-} 
\usepackage{fancyhdr}
\usepackage{amsmath,xcolor}
\usepackage[inline]{enumitem}
\usepackage{amsmath} 
\usepackage{arydshln}
\usepackage{mathtools}


\makeatletter
\newcommand{\xRightarrow}[2][]{\ext@arrow 0359\Rightarrowfill@{#1}{#2}}
\makeatother

\newenvironment{bmatrixcolor}[1][red]
  {\colorlet{savethecolor}{.}\colorlet{bracecolor}{#1}%
    \color{bracecolor}\left[\color{savethecolor}\begin{matrix}}
  {\end{matrix}\color{bracecolor}\right]}




\definecolor{bubbles}{rgb}{0.91, 1.0, 1.0}
\pagecolor{black}
\color{white}
\definecolor{mediumspringgreen}{rgb}{0.0, 0.98, 0.6}
\definecolor{persianrose}{rgb}{1.0, 0.16, 0.64}
\definecolor{cadmiumyellow}{rgb}{1.0, 0.96, 0.0}
\definecolor{cyan(process)}{rgb}{0.0, 0.72, 0.92}
\definecolor{orange-red}{rgb}{1.0, 0.27, 0.0}
\definecolor{antiquecadmiumyellow}{rgb}{0.98, 0.92, 0.84} %Colores de matrices.
\definecolor{apricot}{rgb}{0.98, 0.81, 0.69}
\definecolor{babypink}{rgb}{0.96, 0.76, 0.76}
\definecolor{guppiegreen}{rgb}{0.0, 1.0, 0.5}
\definecolor{lava}{rgb}{0.81, 0.06, 0.13}
\definecolor{maize}{rgb}{0.98, 0.93, 0.37}
\definecolor{redwood}{rgb}{0.67, 0.31, 0.32}
\definecolor{rosybrown}{rgb}{0.74, 0.56, 0.56}


\pagestyle{fancy} 
\fancyhead{}\renewcommand{\headrulewidth}{0pt} 
\fancyfoot[C]{} 
\fancyfoot[R]{\thepage} 
\newcommand{\note}[1]{\marginpar{\scriptsize \textcolor{red}{#1}}} 
\begin{document}
\fancyhead[C]{}
\begin{minipage}{0.295\textwidth} 
\raggedright
Equipo\\    
\footnotesize 
\colorbox[rgb]{0.5, 1.0, 0.83}{\textcolor{black}{Aguilar Valenzuela Montserrat}}
\\\colorbox[rgb]{0.94, 0.86, 0.51}{\textcolor{black}{Cruz González Irvin Javier}}
\\\colorbox[rgb]{0.99, 0.76, 0.8}{\textcolor{black}{Murrillo Rosas Estefania}}
\textcolor[rgb]{0.91, 1.0, 1.0}{\medskip\hrule}
\end{minipage}
\begin{minipage}{0.4\textwidth} 
\centering 
\large 
\textbf{Matemáticas para las Ciencias Aplicadas II}\\ 
\normalsize 
Tarea 5\\
\end{minipage}
\begin{minipage}{0.295\textwidth} 
\raggedleft
\today\\ 
\footnotesize
mnts94@ciencias.unam.mx
1rv1n@ciencias.unam.mx
hollymol7@ciencias.unam.mx 
\textcolor[rgb]{0.91, 1.0, 1.0}{\medskip\hrule}
\end{minipage}

%colorcitos
\definecolor{iris}{rgb}{0.35, 0.31, 0.81}
\definecolor{outrageousorange}{rgb}{1.0, 0.43, 0.29}
\definecolor{sangria}{rgb}{0.57, 0.0, 0.04}
\definecolor{oceanboatblue}{rgb}{0.0, 0.47, 0.75}

\begin{enumerate}

                                                %EJERCICIO 1 :)
    
    \item Escribe los siguientes vectores como una combinación lineal del conjunto $S\{ (1,2-2),(2,-1,1)\}$
    
    \begin{enumerate}
          \item   \textbf{z}=(-4,-3,3)
          \item   \textbf{v}=(-2,-6,6)
          \item   \textbf{w}=(-1,-22,22)
          \item   \textbf{u}=(1,-5,-5)


    \end{enumerate}
    {\color{oceanboatblue} \rule{\linewidth}{0.5mm} }

    \begin{enumerate}

        \item Encontrar los escalares ($c_{1},c_{2}$) tales que $(-4,-3,3)=\\\\c_{1}(1,2,-2)+c_{2}(2,-1,1)$\\
        
        $(-4,-3,3)=(c_{1}+2c_{2},2c_{1}-c_{2},-2c_{1}+c_{2})$}\\\\ \hspace{.5cm}{\textit{\textcolor{pink}{Resolviendo con Eliminación Gaussiana}}}}\\

            \left\lbrace
            \begin{array}{ll}
            \textup{$c_{1}+2c_{2}=-4$ } \\\\
            \textup{$2c_{1}-c_{2}=-3$}\\\\
            \textup{$-2c_{1}+c_{2}=3$}
            \end{array} 
            \right.\hspace{.2cm}{$\xRightarrow$}\hspace{.2cm}{ $\begin{bmatrixcolor}[bubbles]
                1 & 2 & -4   \\
                2 & -1 & -3  \\
                -2 & 1 & 3         
            \end{bmatrixcolor}$} \hspace{.2cm}{$\xRightarrow{\mathit{-2F_{1}+F_{2}}}}$} \hspace{.2cm}{ $\begin{bmatrixcolor}[bubbles]
                1 & 2 & -4   \\
                0 & -5 & 5  \\
                -2 & 1 & 3         
            \end{bmatrixcolor}$} \hspace{.2cm}{$\xRightarrow{\mathit{2F_{1}+F_{3}}}}}$}\\\\

            $\begin{bmatrixcolor}[bubbles]
                1 & 2 & -4   \\
                0 & -5 & 5  \\
                0 & -5 & -5         
            \end{bmatrixcolor}$ \hspace{.2cm}{$\xRightarrow{\mathit{-\frac{1}{5}F_{2}}}}$}\hspace{.2cm}{ $\begin{bmatrixcolor}[bubbles]
                1 & 2 & -2   \\
                0 & -1 & 1  \\
                0 & 5 & -5         
            \end{bmatrixcolor}$}\hspace{.2cm}{$\xRightarrow{\mathit{-5F_{2}+F_{3}}}$}\hspace{.2cm}{ $\begin{bmatrixcolor}[bubbles]
                1 & 2 & -4   \\
                0 & 1 & -1  \\
                0 & 0 & 0         
            \end{bmatrixcolor}$} = \\\\

            \textcolor{green}{$c_{1}+2c_{2}=-4$ \hspace{.5cm}{$c_{1}=-4-2(-1)$} \hspace{.5cm}{c_{1}=-4+2}}\\\\
            \textcolor{green}{$c_{2}=-1$}\\
            \textcolor{green}{$c_{1}=-2 \therefore Z $} es una combinación lineal de $S.$}\\


        
            \item Encontrar los escalares ($c_{1},c_{2}$) tales que $(-2,-6,6)=\\\\c_{1}(1,2,-2)+c_{2}(2,-1,1)$\\
        
            $(-2,-6,6)=(c_{1}+2c_{2},2c_{1}-c_{2},-2c_{1}+c_{2})$}\\\\ \hspace{.5cm}{\textit{\textcolor{pink}{Resolviendo con Gauss-Jordan}}}\\

            \left\lbrace
            \begin{array}{ll}
            \textup{$c_{1}+2c_{2}=-2$ } \\\\
            \textup{$2c_{1}-c_{2}=-6$}\\\\
            \textup{$-2c_{1}+c_{2}=6$}
            \end{array}\hspace{.2cm}{$\xRightarrow$}\hspace{.2cm}{ $\begin{bmatrixcolor}[red]
                1 & 2 & -2   \\
                2 & -1 & -6  \\
                -2 & 1 & 6         
            \end{bmatrixcolor}$} \hspace{.2cm}{$\xRightarrow{\mathit{-2F_{1}+F_{2}}}$} \hspace{.2cm}{ $\begin{bmatrixcolor}[red]
                1 & 2 & -2   \\
                0 & -5 & -2  \\
                -2 & 1 & 6        
            \end{bmatrixcolor}$} \hspace{.2cm}{$\xRightarrow{\mathit{2F_{1}+F_{3}}}$}\\\\

            $\begin{bmatrixcolor}[red]
                1 & 2 & -2   \\
                0 & -5 & -2  \\
                0 & -5 & 2         
            \end{bmatrixcolor}$ \hspace{.2cm}{$\xRightarrow{\mathit{-\frac{1}{5}F_{2}}}$}\hspace{.2cm}{ $\begin{bmatrixcolor}[red]
                1 & 2 & -2   \\\\
                0 & 1 & \frac{2}{5}  \\\\
                0 & 5 & 2         
            \end{bmatrixcolor}$}\hspace{.2cm}{$\xRightarrow{\mathit{-5F_{2}+F_{3}}}}$}\hspace{.2cm}{ $\begin{bmatrixcolor}[red]
                1 & 2 & -2   \\\\
                0 & 1 & \frac{2}{5}  \\\\
                0 & 0 & 0         
            \end{bmatrixcolor}$} \hspace{.2cm}{$\xRightarrow{\mathit{-2F_{2}+F_{1}}}}$}\hspace{.2cm}{ $\begin{bmatrixcolor}[red]
                1 & 0 & -\frac{14}{5}   \\\\
                0 & 1 & \frac{2}{5}  \\\\
                0 & 0 & 0         
            \end{bmatrixcolor}$} \\\\

            \textcolor{green}{$c_{1}=-\frac{14}{5},\hspace{.4cm}{c_{2}=\frac{2}{5}}$\hspace{.4cm} {conjunto solución $\therefore$ $V$ es combinación lineal de $S$}}\\
   
   
            \item Encontrar los escalares ($c_{1},c_{2}$) tales que $(1,-22,22)\\\\=c_{1}(1,2,-2)+c_{2}(2,-1,1)$\\
        
            $(1,-22,22)=(c_{1}+2c_{2},2c_{1}-c_{2},-2c_{1}+c_{2})$}\\\\ \hspace{.5cm}{\textit{\textcolor{pink}{Resolviendo con Gauss-Jordan}}}\\

            \left\lbrace
            \begin{array}{ll}
            \textup{$c_{1}+2c_{2}=1$ } \\\\
            \textup{$2c_{1}-c_{2}=-22$}\\\\
            \textup{$-2c_{1}+c_{2}=22$}
            \end{array}\hspace{.2cm}{$\xRightarrow$}\hspace{.2cm}{ $\begin{bmatrixcolor}[cyan]
                1 & 2 & 1   \\
                2 & -1 & -22  \\
                -2 & 1 & 22         
            \end{bmatrixcolor}$} \hspace{.2cm}{$\xRightarrow{\mathit{-2F_{1}+F_{2}}}}$} \hspace{.2cm}{ $\begin{bmatrixcolor}[cyan]
                1 & 2 & 1   \\
                0 & -5 & -24  \\
                -2 & 1 & -22        
            \end{bmatrixcolor}$} \hspace{.2cm}{$\xRightarrow{\mathit{2F_{1}+F_{3}}}}}$}\\\\

            $\begin{bmatrixcolor}[cyan]
                1 & 2 & -1   \\
                0 & -5 & -24  \\
                0 & -5 & 24         
            \end{bmatrixcolor}$ \hspace{.2cm}{$\xRightarrow{\mathit{-\frac{1}{5}F_{2}}}}}$}\hspace{.2cm}{ $\begin{bmatrixcolor}[cyan]
                1 & 2 & -1  \\\\
                0 & 1 & \frac{24}{5}  \\\\
                0 & 5 & 24         
            \end{bmatrixcolor}$}\hspace{.2cm}{$\xRightarrow{\mathit{-5F_{2}+F_{3}}}}}$}\hspace{.2cm}{ $\begin{bmatrixcolor}[cyan]
                1 & 2 & 1   \\\\
                0 & 1 & \frac{24}{5}  \\\\
                0 & 0 & 0         
            \end{bmatrixcolor}$} \hspace{.2cm}{$\xRightarrow{\mathit{-2F_{2}+F_{1}}}}}$}\hspace{.2cm}{ $\begin{bmatrixcolor}[cyan]
                1 & 0 & -\frac{43}{5}   \\\\
                0 & 1 & \frac{24}{5}  \\\\
                0 & 0 & 0         
            \end{bmatrixcolor}$} \\\\

            \textcolor{green}{$c_{1}=-\frac{43}{5},\hspace{.4cm}{c_{2}=\frac{24}{5}}}$\hspace{.4cm} {conjunto solución $\therefore$ $W$ es combinación lineal de $S$}}\\


       \item  Encontrar los escalares ($c_{1},c_{2}$) tales que $(1,-5,-5)\\\\=c_{1}(1,2,-2)+c_{2}(2,-1,1)$\\
        
            $(1,-5,-5)=(c_{1}+2c_{2},2c_{1}-c_{2},-2c_{1}+c_{2})$}\\\\ \hspace{.5cm}{\textit{\textcolor{pink}{Resolviendo con Gauss-Jordan}}}\\

            \left\lbrace
            \begin{array}{ll}
            \textup{$c_{1}+2c_{2}=1$ } \\\\
            \textup{$2c_{1}-c_{2}=-5$}\\\\
            \textup{$-2c_{1}+c_{2}=-5$}
            \end{array}\hspace{.2cm}{$\xRightarrow$}\hspace{.2cm}{ $\begin{bmatrixcolor}[orange]
                1 & 2 & 1   \\
                2 & -1 & -5  \\
                -2 & 1 & 5         
            \end{bmatrixcolor}$} \hspace{.2cm}{$\xRightarrow{\mathit{-2F_{1}+F_{2}}}}}$} \hspace{.2cm}{ $\begin{bmatrixcolor}[orange]
                1 & 2 & 1   \\
                0 & -5 & -7  \\
                -2 & 1 & -5        
            \end{bmatrixcolor}$} \hspace{.2cm}{$\xRightarrow{\mathit{2F_{1}+F_{3}}}}$}\\\\

            $\begin{bmatrixcolor}[orange]
                1 & 2 & -1   \\
                0 & -5 & -7  \\
                0 & -5 & -3         
            \end{bmatrixcolor}$ \hspace{.2cm}{$\xRightarrow{\mathit{-\frac{1}{5}F_{2}}}}}$}\hspace{.2cm}{ $\begin{bmatrixcolor}[orange]
                1 & 2 & 1  \\\\
                0 & 1 & \frac{7}{5}  \\\\
                0 & 5 & -3         
            \end{bmatrixcolor}$}\hspace{.2cm}{$\xRightarrow{\mathit{-5F_{2}+F_{3}}}}}$}\hspace{.2cm}{ $\begin{bmatrixcolor}[orange]
                1 & 2 & 1   \\\\
                0 & 1 & \frac{7}{5}  \\\\
                0 & 0 & -10         
            \end{bmatrixcolor}$} \\\\

            \textcolor{green}{$\therefore$ $U$ NO es combinación lineal de $S$,el sistema es inconsistente,no se puede solucionar.}}  

    \end{enumerate}

\newpage    
                                                %EJERCICIO 2 :|
    \item Determina si los siguientes conjuntos son linealmente independientes o linealmente dependientes:
    
    \begin{enumerate}
        \item $S = \{(−2, 1, 3),(2, 9, −3),(2, 3, −3)\}$
        \item $S = \{(−4, −3, 4),(1, −2, 3),(6, 0, 0)\}$
    \end{enumerate}

    {\color{oceanboatblue} \rule{\linewidth}{0.5mm} }

    \textcolor{pink}{Ecuación vectorial.\\$c_{1}\vec{v_{1}}+c_{2}\vec{v_{2}}+...+c_{n}\vec{v_{n}}$}

    
    \begin{enumerate}
        
        \item Formando Ec vectorial\\\\
            $c_{1}\vec{v_{1}}+c_{2}\vec{v_{2}}+c_{3}\vec{v_{3}}=\vec{0}}$ \\\\
            $c_{1}(-2,1,3)+c_{2}(2,9-3)+c_{3}(2,3,-3)$ \\\\
            $(-2c_{1}+2c_{2}+2c_{3},c_{1}+9c_{2}+3c_{3},3c_{1}-3c_{2}-3c_{3})=(0,0,0)$\\\\
            \left\lbrace
            \begin{array}{ll}
            \textup{$-2c_{1}+2c_{2}+2c_{3}=0$ } \\\\
            \textup{$c_{1}+9c_{2}+3c_{3}=0$}\\\\
            \textup{$3c_{1}-3c_{2}-3c_{3}=0$}
            \end{array}\hspace{.2cm}{$\xRightarrow$}\hspace{.2cm}{ $\begin{bmatrixcolor}[cadmiumyellow]
                -2 & 2 & 2 & 0   \\
                1 & 9 & 3 & 0 \\
                3 & -3 & -3  & 0       
            \end{bmatrixcolor}$}\hspace{.2cm}{$\xRightarrow{\mathit{-\frac{1}{2}F_{1}}}$} \hspace{.2cm}{ $\begin{bmatrixcolor}[cadmiumyellow]
                1 & -1 & 1 &  0 \\
                1 & 9 & 3 & 0  \\
                3 & -3 & -3  & 0      
            \end{bmatrixcolor}$}\hspace{.2cm}{$\xRightarrow{\mathit{-1(F_{1})+F_{2}}}$}\\\\

            $\begin{bmatrixcolor}[cadmiumyellow]
                1 & -1 & 1 &  0 \\
                1 & 10 & 4 & 0  \\
                3 & -3 & -3  & 0      
            \end{bmatrixcolor}$\hspace{.2cm}{$\xRightarrow{\mathit{-3F_{1}+F_{3}}}$} \hspace{.2cm}{ $\begin{bmatrixcolor}[cadmiumyellow]
                1 & -1 & 1 &  0 \\
                1 & 10 & 4 & 0  \\
                0 & 0 & 0  & 0      
            \end{bmatrixcolor}$} \\\\ 
            
            \textcolor{green}{Solución trivial $\therefore S$ es linealmente dependiente.}

            \item Formando Ec vectorial.\\
            
            $c_{1}(-4,-3,4)+c_{2}(1,-2,3)+c_{3}(6,0,0)$\\\\
            $(-4c_{1}+c_{2}+6c_{3},-3c_{1}-2c_{2},4c_{1}+3c_{2})=(0,0,0)$\\\\
            \left\lbrace
            \begin{array}{ll}
            \textup{$-4c_{1}+c_{2}+6c_{3}=0$ } \\\\
            \textup{$-3c_{1}-2c_{2}=0$}\\\\
            \textup{$4c_{1}+3c_{2}=0$}
            \end{array}\hspace{.2cm}{$\xRightarrow$}\hspace{.2cm}{ $\begin{bmatrixcolor}[persianrose]
                -4 & 1 & 6 & 0   \\
                -3 & -2 & 0 & 0 \\
                4 & 3 & 0  & 0       
            \end{bmatrixcolor}$}\hspace{.2cm}{$\xRightarrow{\mathit{-\frac{1}{4}F_{1}}}$} \hspace{.2cm}{ $\begin{bmatrixcolor}[persianrose]
                1 & \frac{1}{4} & -\frac{3}{2} &  0 \\\\
                -3 & -2 & 0 & 0  \\\\
                4 & 3 & 0  & 0      
            \end{bmatrixcolor}$}\hspace{.2cm}{$\xRightarrow{\mathit{3F_{1}+F_{2}}}$}\\\\

            $\begin{bmatrixcolor}[persianrose]
                1 & -\frac{1}{4} & -\frac{3}{2} &  0 \\\\
                0 & -\frac{11}{4} & -\frac{9}{2} & 0  \\\\
                4 & 3 & 0  & 0      
            \end{bmatrixcolor}$\hspace{.2cm}{$\xRightarrow{\mathit{-4F_{1}+F_{3}}}$} \hspace{.2cm}{  $\begin{bmatrixcolor}[persianrose]
                1 & -\frac{1}{4} & -\frac{3}{2} &  0 \\\\
                0 & -\frac{11}{4} & -\frac{9}{2} & 0  \\\\
                0 & 4 & 6  & 0      
            \end{bmatrixcolor}$} \hspace{.2cm}{$\xRightarrow{\mathit{-\frac{4}{11}F_{2}}}$} \hspace{.2cm}{  $\begin{bmatrixcolor}[persianrose]
                1 & -\frac{1}{4} & -\frac{3}{2} &  0 \\\\
                0 & 1 & \frac{18}{11} & 0  \\\\
                0 & 4 & 6  & 0      
            \end{bmatrixcolor}$} \hspace{.2cm}{$\xRightarrow{\mathit{-4F_{2}+F_{3}}}$}\\\\ 

            $\begin{bmatrixcolor}[persianrose]
                1 & -\frac{1}{4} & -\frac{3}{2} &  0 \\\\
                0 & 1 & \frac{18}{11} & 0  \\\\
                0 & 0 & -\frac{6}{11}  & 0      
            \end{bmatrixcolor}$\hspace{.2cm}{$\xRightarrow{\mathit{\frac{1}{4}F_{2}+F_{1}}}$} \hspace{.2cm}{  $\begin{bmatrixcolor}[persianrose]
                1 & 0 & -\frac{12}{11} &  0 \\\\
                0 & 1 & \frac{18}{11} & 0  \\\\
                0 & 4 & -\frac{6}{11}  & 0      
            \end{bmatrixcolor}$}\hspace{.2cm}{$\xRightarrow{\mathit{-\frac{6}{11}F_{3}}}$} \hspace{.2cm}{  $\begin{bmatrixcolor}[persianrose]
                1 & 0 & -\frac{12}{11} &  0 \\\\
                0 & 1 & \frac{18}{11} & 0  \\\\
                0 & 0 & 1  & 0      
            \end{bmatrixcolor}$} \hspace{.2cm}{$\xRightarrow{\mathit{-\frac{18}{11}F_{3}+F_{2}}}$}\\\\

            $\begin{bmatrixcolor}[persianrose]
                1 & 0 & -\frac{12}{11} &  0 \\\\
                0 & 1 & 0 & 0  \\\\
                0 & 0 & 1  & 0      
            \end{bmatrixcolor}$\hspace{.2cm}{$\xRightarrow$} \hspace{.1cm}{  $\begin{bmatrixcolor}[persianrose]
                1 & 0 & 0 &  0 \\\\
                0 & 1 & 0 & 0  \\\\
                0 & 0 & 1  & 0      
            \end{bmatrixcolor}$} 

            \textcolor{green}{$c_{1}=0\hspace{.3cm}{c_{2}=0\hspace{.3cm}{c_{3}=0}} \therefore$ el sistema es linealmente independiente}


    \end{enumerate}



                                                %EJERCICIO 3 ;(
     \item Determina si los siguientes conjuntos generan $\mathds{R}^2$. Si no lo genera, da una descripción geométrica del espacio que genera.
     
     \begin{enumerate}
         \item $S = \{(−1, 1),(3, 1)\}$
         \item $S = \{(−1, 2),(2, −4)\}$
         \item $S = \{(1, 3),(−2, −6),(4, 12)\}$
     \end{enumerate}

     {\color{oceanboatblue} \rule{\linewidth}{0.5mm} }

     \begin{enumerate}
         
        \item ¿$S$ genera $\mathds{R^{2}}$?\\
        
        $\vec{v}=(u_{1},u_{2})$\\\\ $c_{1}(-1,1)+c_{2}(3,1)=u_{1},u_{2}$\\\\
        $(-c_{1}+3c_{2},c_{1}+c_{2})=(u_{1},u_{2})$\\

        \left\lbrace
            \begin{array}{ll}
            \textup{$-c_{1}+3c_{2}=u_{1}$ } \\\\
            \textup{$-3c_{1}-2c_{2}=u_{2}$}
            \end{array}\hspace{.2cm}{$\xRightarrow$}\hspace{.2cm}{ $\begin{bmatrixcolor}[green]
                -1 & 3 & u_{1} &   \\
                1 & 1 & u_{2} &         
            \end{bmatrixcolor}$} \hspace{.2cm}{$\xRightarrow{\mathit{F_{1} \rightleftarrows F_{2}}}$}\hspace{.2cm}{\hspace{.2cm}{ $\begin{bmatrixcolor}[green]
                1 & 1 & u_{2} &   \\
                -1 & 3 & u_{1} &         
            \end{bmatrixcolor}$}}\hspace{.2cm}{$\xRightarrow{\mathit{F_{1}+ F_{2}}}$}\hspace{.2cm}{\hspace{.1cm}{ $\begin{bmatrixcolor}[green]
                1 & 1 & u_{2} &   \\
                0 & 4 & u_{1} &         
            \end{bmatrixcolor}$}}\\\\

            $\xRightarrow{\mathit{\frac{1}{4}F_{2}}}$\hspace{.2cm} {$\begin{bmatrixcolor}[green]
                1 & 1 & u_{2} &   \\
                0 & 4 & u_{1} &         
            \end{bmatrixcolor}$}\hspace{.2cm}{$\xRightarrow{\mathit{-F_{2}+ F_{1}}}$}\hspace{.2cm}{\hspace{.1cm}{ $\begin{bmatrixcolor}[green]
                1 & 1 & -\frac{u_{1}+u_{2}}{4}+u_{2} & \\\\
                0 & 4 & \frac{u_{1}+u_{2}}{4}&         
            \end{bmatrixcolor}$}}\hspace{.3cm}{=}{$\begin{bmatrixcolor}[green]
                1 & 0 & \frac{3u_{2}-u_{1}}{4} & \\\\
                0 & 1 & \frac{u_{1}+u_{2}}{4}&         
            \end{bmatrixcolor}$}\\\\

            \textcolor{green}{$c_{1}=\frac{3u_{2}-u_{1}}{4} \hspace{.4cm}{,c_{2}=\frac{u_{1}+u_{2}}{4}}$}\\\\

            ¿$S$ es linealmente independiente?\\\\
            $c_{1}(-1,1)+c_{2(3,1)}$ \hspace{4.5cm}{De $2$}\hspace{2cm}{ $2c_{2}=0$}\\\\
            $(-c_{1}+3c_{2},c_{1}+c_{2})=(0,0,0)$\hspace{2.6cm}{$c_{1}=-c_{2}$}\hspace{1.5cm}{ $c_{2}=0$}\\\\
            $-c_{1}+3c_{2}=0$\hspace{.4cm}{\textcolor{yellow}{...(1)}}\hspace{4cm}{Sust en \textcolor{yellow}{(1)}}\hspace{1.4cm}{\textcolor{green}{ solución}}\\\\
            $-c_{1}+c_{2}=0$\hspace{.6cm}{\textcolor{yellow}{...(2)}\hspace{3.7cm}{ $-c_{2}+3c_{2}=0$}}\hspace{1cm}{\textcolor{green}{$c_{1}=0$ y $c_{2}=0$}}\\\\
           \textcolor{green}{ $\therefore S$ es linealmente independiente ya que genera $\mathds{R}^{2}}$}\\

           \item ¿ S genera a $\mathds{R}^2$ ?\\ \\ $\vec{v}=(u_{1},u_{2})$\\\\ $C_{1}(-1,2)+c_{2}(2,-4)=u_{1}u_{2}$\\\\
           $(-c_{1}+2c_{2},2c_{1}-4c_{2})=(u_{1},u_{2})$ \hspace{1cm}{\left\lbrace
           \begin{array}{ll}
           \textup{$-c_{1}+2c_{2}=u_{1}$ } \\\\
           \textup{$2c_{1}-4c_{2}=u_{2}$}
           \end{array}}
\newpage
           $\xRightarrow$}\hspace{.2cm}{ $\begin{bmatrixcolor}[lava]
            -1 & 2 & u_{1} &   \\
            2 & 4& u_{2} &         
        \end{bmatrixcolor}}$\hspace{.2cm}{$\xRightarrow{\mathit{F_{1} \rightleftarrows F_{2}}}$}\hspace{.2cm}{\hspace{.2cm}{ $\begin{bmatrixcolor}[lava]
            2 & -4 & u_{2} &   \\
            -1 & 2 & u_{1} &         
        \end{bmatrixcolor}$}}\hspace{.2cm}{$\xRightarrow{\mathit{\frac{1}{2}F_{1}}}$}\hspace{.2cm}{\hspace{.2cm}{ $\begin{bmatrixcolor}[lava]
            1 & 2 &\frac{u_{2}}{2} &   \\\\
            -1 & 2 & u_{1} &         
        \end{bmatrixcolor}$}}\hspace{.2cm}{$\xRightarrow{\mathit{F_{1}+F_{2}}}$}\\\\
         $\begin{bmatrixcolor}[lava]
            1 & -2 & \frac{u_{2}}{2} &   \\\\
            0 & 0 & -2u_{1}+\frac{u_{2}}{2} &         
        \end{bmatrixcolor}$ \hspace{1cm}{\textcolor{green}{$\therefore S$ no genera $\mathds{R}^2}$}\\\\
        \begin{center}
            \includegraphics[width=7cm, height=5cm ]{ejercicio3(b).png}

        \end{center}
        
        \item ¿ S genera a $\mathds{R}^2$ ?\\ \\ $\vec{v}=(u_{1},u_{2})$\\\\ $c_{1}(1,3)+c_{2}(-2,-6),c_{3}(4,12)=u_{1}u_{2}$\\\\
        $(c_{1}-2c_{2}+4c_{3},3c_{1}-6c_{2}+12c_{3})=(u_{1},u_{2})$\\\\
        \left\lbrace
           \begin{array}{ll}
           \textup{$c_{1}-2c_{2}+4c_{3}=u_{1}$ } \\\\
           \textup{$3c_{1}-6c_{2}+12c_{3}=u_{2}$}
           \end{array} \hspace{.2cm}{$ \xRightarrow $}{\hspace{.2cm}{\hspace{.2cm}{ $\begin{bmatrixcolor}[maize]
            1 & -2 & 4 & u_{1} &   \\
            3 & -6 & 12 & u_{2}} &         
        \end{bmatrixcolor}$}} \hspace{.2cm}{$\xRightarrow{\mathit{-3F_{1}+F_{2}}}$}\hspace{.2cm}{\hspace{.2cm}{ $\begin{bmatrixcolor}[maize]
            1 & -2 & 4 & u_{1}   \\\\
            0 & 0 & 0 & 3u_{1}+u_{2} &         
        \end{bmatrixcolor}$}} \\\\

        \textcolor{green}{$\therefore S$ no genera a $\mathds{R}^2$} 

        \begin{center}
            \includegraphics[width=7cm, height=5cm ]{ejercicio3(c).png}

        \end{center}


     \end{enumerate}


                                          %EJERCICIO 4:(    
     \item Determina si los siguientes sistemas son consistentes. Si es así, escribe la solución en la forma $x=x_{p}+x_{h}$,donde $x_{p}$ es una solución
            particular de $Ax=b$ y $x_{h}$ es una solución de $Ax=0$.

     \begin{enumerate*}
         \item $x-4y=17$  \hspace{3cm}{\hspace*{.5cm}{\item $3x-8y+4z=19$}}\\\\
        \hspace*{1cm}{$3x-12y=51$}\hspace{2.5cm}{\hspace*{.5cm}{ $-6y+2z+4w=5$}}\\\\
        \hspace*{.7cm}{$-2x+8y=-34$}\hspace{2.5cm}{\hspace*{.5cm}{ $5x+22z+w=29$}}\\\\  
        \hspace*{6.8cm}{$x-2y+2z=8$}
     \end{enumerate*}   

     {\color{oceanboatblue} \rule{\linewidth}{0.5mm} }

     \begin{enumerate}
         
        \item  $\begin{bmatrixcolor}[orange]
            1 & 4 & 17   \\
            3 & -12 & 51  \\
            -2 & 8 & -34         
        \end{bmatrixcolor}$\hspace{.2cm}{$\xRightarrow{\mathit{-3F_{1}+F_{2}}}$} \hspace{.2cm}{ $\begin{bmatrixcolor}[orange]
            1 & 4 & 17   \\
            0 & -24 & 0  \\
            -2 & 8 & -34        
        \end{bmatrixcolor}$}\hspace{.2cm}{$\xRightarrow{\mathit{2F_{1}+F_{3}}}$} \hspace{.2cm}{ $\begin{bmatrixcolor}[orange]
            1 & 4 & 17   \\
            0 & -24 & 0  \\
            0 & 16 & 0        
        \end{bmatrixcolor}$}\hspace{.2cm}{$\xRightarrow{\mathit{-\frac{1}{24}F_{2}}}}$}\\\\ 
        
        $\begin{bmatrixcolor}[orange]
            1 & 4 & 17   \\
            0 & -1 & 0  \\
            0 & 16 & 0        
        \end{bmatrixcolor}$\hspace{.2cm}{$\xRightarrow{\mathit{-4F_{2}+F_{1}}}$} \hspace{.2cm}{ $\begin{bmatrixcolor}[orange]
            1 & 0 & 17   \\
            0 & 1 & 0  \\
            0 & 16 & 0        
        \end{bmatrixcolor}$}\hspace{.2cm}{$\xRightarrow{\mathit{-16F_{2}+F_{3}}}$} \hspace{.2cm}{ $\begin{bmatrixcolor}[orange]
            1 & 0 & 17   \\
            0 & 1 & 0  \\
            0 & 0 & 0        
        \end{bmatrixcolor}$}\\\\

        $\textcolor{green}{x=17$}\\
        $\textcolor{green}{y=0 \hspace{.4cm}{\therefore$ el sistema es consistente}}\\\\


        \item $\begin{bmatrixcolor}[cyan]
            0 & 3 & -8 & 4 & 19   \\
            4 & 0 & -6 & 2 & 5   \\
            1 & 5 & -0 & 22 & 29  \\
            0 & 1 & -2 & 2 & 8         
        \end{bmatrixcolor}$\hspace{.2cm}{$F_{3} \rightleftarrows F_{1}$} \hspace{.2cm}{$\begin{bmatrixcolor}[cyan]
            1 & 5 & 0 & 22 & 29   \\
            4 & 0 & -6 & 2 & 5   \\
            0 & 3 & -8 & 4 & 19  \\
            0 & 1 & -2 & 2 & 8         
        \end{bmatrixcolor}$ }\hspace{.2cm}{$\xRightarrow{\mathit{4F_{1}+F_{2}}}$} \hspace{.2cm}{$\begin{bmatrixcolor}[cyan]
            1 & 5 & 0 & 22 & 29   \\
            0 & 20 & -6 & -86 & -111   \\
            0 & 3 & -8 & 4 & 19  \\
            0 & 1 & -2 & 2 & 8         
        \end{bmatrixcolor}$ } \\\\

        $\xRightarrow{\mathit{F_{4} \rightleftarrows F_{2}}}}$ $\begin{bmatrixcolor}[cyan]
            1 & 5 & 0 & 22 & 29   \\ 
            4 & 1 & -2 & 2 & 8   \\
            1 & 3 & -8 & 4 & 19  \\
            0 & 20 & -6 & -86 & -111         
        \end{bmatrixcolor}$ \hspace{.2cm}{$\xRightarrow{\mathit{-3F_{2}+F_{3}}}$} \hspace{.2cm}{$\begin{bmatrixcolor}[cyan]
            1 & 5 & 0 & 22 & 29   \\
            0 & 1 & -2 & 2 & 8   \\
            0 & 0 & -2 & -2 & -5  \\
            0 & 20 & -6 & -86 & -111         
        \end{bmatrixcolor}$ }\hspace{.2cm}{$\xRightarrow{\mathit{-20F_{2}+F_{4}}}$}\\\\

        $\begin{bmatrixcolor}[cyan]
            1 & 5 & 0 & 22 & 29   \\
            0 & 1 & -2 & 2 & 8   \\
            0 & 0 & -2 & -2 & -5  \\
            0 & 0 & -34 & -126 & -171         
        \end{bmatrixcolor}$ .... \hspace{.3cm}{  $\begin{bmatrixcolor}[cyan]
            1 & 0 & 0 & 2 & -36   \\
            0 & 1 & 0 & 4 & 13   \\
            0 & 0 & 1 & 1 & 2.5  \\
            0 & 0 & 0 & 0 & 164         
        \end{bmatrixcolor}$ } \\\\ 
        
        \textcolor{green}{{$ 0 \neq 164 \therefore $ El sistema es inconsistente.}}


     \end{enumerate}






     \item Encuentra \textbf{una base del espacio generado por los renglones, el rango y el espacio nulo} de las siguientes
     matrices.

     \begin{enumerate*}
            \item  $A=$ $\begin{bmatrixcolor}[iris]
                2  & 5 \\
                -2 & -5 \\
                -6 & -15
            \end{bmatrixcolor}$\hspace*{1cm}{\hspace*{1cm} {\item  $B=$ $\begin{bmatrixcolor}[outrageousorange]
                2 & -3 & 1\\
               5 & 10 & 6 \\
                8 & -7 & 5
            \end{bmatrixcolor}$\hspace*{1cm}{\hspace*{1cm} \item  $C=$ $\begin{bmatrixcolor}[outrageousorange]
                1 & 4 & 2\\
               0 & 0 & 1 
            \end{bmatrixcolor}$\\\\\\\\

            \hspace*{4.5cm}{\hspace*{1cm}\item $D=$ $\begin{bmatrixcolor}[sangria]
                5  & 2 \\
                3 & -1 \\
                2 & 1
            \end{bmatrixcolor}$}}
     \end{enumerate*}\\

     {\color{oceanboatblue} \rule{\linewidth}{0.5mm} }

     \begin{enumerate}
     
     \item \\ 
    $\xRightarrow{\mathit{\frac{1}{2}R_{1}}}$}\hspace{.2cm}{\hspace{.1cm}{ $\begin{bmatrixcolor}[iris]
        1 & \frac{5}{2}& \\\\
        -2 & -5  & \\\\
        -6 & -15          
    \end{bmatrixcolor}$\hspace{.2cm}{$\xRightarrow{\mathit{2R_{1}+R_{2}}}${\hspace{.1cm}{ $\begin{bmatrixcolor}[iris]
        1 & \frac{5}{2}& \\\\
        0 & 0  & \\\\
        -6 & -15          
    \end{bmatrixcolor}}}$\hspace{.2cm}{$\xRightarrow{\mathit{6R_{1}+R_{3}}}${\hspace{.1cm}{ $\begin{bmatrixcolor}[iris]
        1 & \frac{5}{2}& \\\\
        0 & 0  & \\\\
        0 & 0          
    \end{bmatrixcolor}}}$ \\\\

    \textcolor{green}{Los vectores renglones no $0$ de $A$\\$w_{1}=(1,\frac{5}{2})}$\\\\
    \textcolor{green}{Rango(A)=1}

    \textcolor{green}{Espacio nulo}} \\

    $\begin{bmatrixcolor}[iris]
        2 & 5 & 0 & \\
        -2 & -5  & 0 & \\
        -6 & -15 & 0           
    \end{bmatrixcolor}$ \hspace{.2cm}{$\xRightarrow{\mathit{3R_{1}+R_{3}}}$ \hspace{.2cm}{$\begin{bmatrixcolor}[iris]
        2 & 5 & 0 & \\
        0 & 0  & 0 & \\
        -6 & -15 & 0           
    \end{bmatrixcolor}$}\hspace{.2cm}{$\xRightarrow{\mathit{\frac{1}{2}R_{1}}}}$ \hspace{.2cm}{$\begin{bmatrixcolor}[iris]
        2 & 5 & 0 & \\
        0 & 0  & 0 & \\
        0 & 0 & 0           
    \end{bmatrixcolor}$}\hspace{.2cm}{$\xRightarrow$ \hspace{.2cm}{$\begin{bmatrixcolor}[iris]
        1 & \frac{5}{2} & 0 & \\
        0 & 0  & 0 & \\
        0 & 0 & 0           
    \end{bmatrixcolor}$} \\

    \textcolor{green}{Sistema Equivalente \\\\ $x_{1}+\frac{5}{2}x_{2}=0 \\\\ x_{1}=-\frac{5}{2}x_{2}\\\\ -\frac{5}{2}x_{2}+\frac{5}{2}x_{2}=0\\\\x_{2}=t \hspace{.5cm}{t \in \mathds{R}}$}\\\\


    \item \\
    
    $\xRightarrow{\mathit{\frac{1}{2}R_{1}}}$}\hspace{.2cm}{\hspace{.1cm}{ $\begin{bmatrixcolor}[outrageousorange]
        1 & -\frac{3}{2}& \frac{1}{2}& \\\\
        5 & 10  & 6 & \\\\
        8 & -7 & 5          
    \end{bmatrixcolor}$\hspace{.2cm}{$\xRightarrow{\mathit{-5R_{1}+R_{2}}}${\hspace{.1cm}{ $\begin{bmatrixcolor}[outrageousorange]
        1 & -\frac{3}{2} & \frac{1}{2} & \\\\
        0 & \frac{35}{2}  & \frac{7}{2} & \\\\
        8 & -7 & 5         
    \end{bmatrixcolor}}}$\hspace{.2cm}{$\xRightarrow{\mathit{-8R_{1}+R_{3}}}}}${\hspace{.1cm}{ $\begin{bmatrixcolor}[outrageousorange]
        1 & -\frac{3}{2} & \frac{1}{2} & \\\\
        0 & \frac{35}{2}  & \frac{7}{2} & \\\\
        0 & 5 & 1          
    \end{bmatrixcolor}}}$ \hspace{.2cm}{$\xRightarrow{\mathit{\frac{2}{35}R_{2}}}}$\\\\\\\\ $\begin{bmatrixcolor}[outrageousorange]
        1 & -\frac{3}{2} & \frac{1}{2} & \\\\
        0 & 1 & \frac{1}{5} & \\\\
        0 & 5 & 1          
    \end{bmatrixcolor}$ \hspace{.2cm}{$\xRightarrow{\mathit{\frac{3}{2}R_{2}+R_{1}}}}}${\hspace{.1cm}{ $\begin{bmatrixcolor}[outrageousorange]
        0 & 0 & \frac{4}{5} & \\\\
        0 & 1  & \frac{1}{5} & \\\\
        0 & 5 & 1          
    \end{bmatrixcolor}}}$\hspace{.2cm}{$\xRightarrow{\mathit{-5R_{2}+R_{3}}}}}${\hspace{.1cm}{ $\begin{bmatrixcolor}[outrageousorange]
        0 & 0 & \frac{4}{5} & \\\\
        0 & 1  & \frac{1}{5} & \\\\
        0 & 0 & 0          
    \end{bmatrixcolor}}}$ \\\\

    \textcolor{green}{Los vectores renglones no $0$ de $B$\\$w_{1}=(0,0,\frac{4}{5})}\hspace{.5cm}{w_{2}=(0,1,\frac{1}{5})}$\\\\
    \textcolor{green}{Rango(B)=2}\\

    \textcolor{green}{Espacio nulo}} \\


    $\xRightarrow{\mathit{\frac{1}{2}R_{1}}}$}\hspace{.2cm}{\hspace{.1cm}{ $\begin{bmatrixcolor}[outrageousorange]
        1 & -\frac{3}{2}& \frac{1}{2}& 0 &  \\\\
        5 & 10  & 6 & 0 & \\\\
        8 & -7 & 5  & 0        
    \end{bmatrixcolor}$\hspace{.2cm}{$\xRightarrow{\mathit{-5R_{1}+R_{2}}}${\hspace{.1cm}{ $\begin{bmatrixcolor}[outrageousorange]
        1 & -\frac{3}{2} & \frac{1}{2} & 0 &\\\\
        0 & \frac{35}{2}  & \frac{7}{2} & 0 & \\\\
        8 & -7 & 5   & 0      
    \end{bmatrixcolor}}}$\hspace{.2cm}{$\xRightarrow{\mathit{-8R_{1}+R_{3}}}}}${\hspace{.1cm}{ $\begin{bmatrixcolor}[outrageousorange]
        1 & -\frac{3}{2} & \frac{1}{2} & 0 &\\\\
        0 & \frac{35}{2}  & \frac{7}{2} & 0 & \\\\
        0 & 5 & 1  & 0        
    \end{bmatrixcolor}}}$ \hspace{.2cm}{$\xRightarrow{\mathit{\frac{2}{35}R_{2}}}}$\\\\\\\\ $\begin{bmatrixcolor}[outrageousorange]
        1 & -\frac{3}{2} & \frac{1}{2} & 0 &\\\\
        0 & 1 & \frac{1}{5} &  0 &\\\\
        0 & 5 & 1  & 0        
    \end{bmatrixcolor}$ \hspace{.2cm}{$\xRightarrow{\mathit{\frac{3}{2}R_{2}+R_{1}}}}}${\hspace{.1cm}{ $\begin{bmatrixcolor}[outrageousorange]
        1 & 0 & \frac{4}{5} & 0 & \\\\
        0 & 1  & \frac{1}{5} &  0 &\\\\
        0 & 5 & 1  & 0        
    \end{bmatrixcolor}}}$\hspace{.2cm}{$\xRightarrow{\mathit{-5R_{2}+R_{3}}}}}${\hspace{.1cm}{ $\begin{bmatrixcolor}[outrageousorange]
        1 & 0 & \frac{4}{5} & 0 & \\\\
        0 & 1  & \frac{1}{5} & 0 &\\\\
        0 & 0 & 0  & 0        
    \end{bmatrixcolor}}}$ \\\\ 

    \textcolor{green}{Sistema Equivalente \\\\ $x_{1}+\frac{5}{2}x_{3}=0 \\\\ x_{3}=t\hspace{.5cm}{t \in \mathds{R}}\\\\x_{1}=-\frac{4}{5}t,\hspace{.5cm}{x_{2}=-\frac{1}{5}t}$}\\\\

    \item \textcolor{green}{Los vectores renglones no $0$ de $B$\\$w_{1}=(1,4,2)}\hspace{.5cm}{w_{2}=(0,0,1)}$\\\\
    \textcolor{green}{Rango(C)=2}\\

    \textcolor{green}{Espacio nulo}} \\

    $\begin{bmatrixcolor}[outrageousorange]
        1 & 4 & 2 & 0 & \\
        0 & 0  & 1 & 0&        
    \end{bmatrixcolor}$ \\\\

    \textcolor{green}{Sistema Equivalente \\\\ $x_{1}+4x_{2}+2x_{3}=0 \\\\ x_{3}=0 \hspace{.2cm}{x_{1}=-4x_{2}}\\\\-4x_{2}+4x_{2}=0$\\\\ $x_{2}=t$ \hspace{.5cm} {$t\in \mathds{R}$} \\\\ nulidad=1}


    \item \\
    
    $\xRightarrow{\mathit{\frac{1}{5}R_{1}}}$}\hspace{.2cm}{\hspace{.1cm}{ $\begin{bmatrixcolor}[sangria]
        1 & \frac{2}{5}& \\\\
        3 & -1  & \\\\
        2 & 1          
    \end{bmatrixcolor}}$\hspace{.2cm}{$\xRightarrow{\mathit{-3R_{1}+R_{2}}}}}${\hspace{.1cm}{ $\begin{bmatrixcolor}[sangria]
        1 & \frac{2}{5}& \\\\
        0 & -\frac{11}{5}  & \\\\
        2 & 1         
    \end{bmatrixcolor}}}$\hspace{.2cm}{$\xRightarrow{\mathit{2R_{1}+R_{3}}}}}${\hspace{.1cm}{ $\begin{bmatrixcolor}[sangria]
        1 & \frac{2}{5}& \\\\
        0 & -\frac{11}{5}  & \\\\
        0 & -\frac{9}{5}         
    \end{bmatrixcolor}}}$ \hspace{.2cm}{$\xRightarrow{\mathit{-\frac{5}{11}R_{2}}}}}${\hspace{.1cm}{ $\begin{bmatrixcolor}[sangria]
        1 & \frac{2}{5}& \\\\
        0 & 1  & \\\\
        0 & -\frac{9}{5}         
    \end{bmatrixcolor}}}$\hspace{.2cm}{$\xRightarrow{\mathit{\frac{9}{5}R_{2}+R_{3}}}}}${\hspace{.1cm}{ $\begin{bmatrixcolor}[sangria]
        1 & \frac{2}{5}& \\\\
        0 & 1  & \\\\
        0 & 0        
    \end{bmatrixcolor}}}$\hspace{.2cm}{$\xRightarrow{\mathit{\frac{2}{5}R_{2}+R_{1}}}}}${\hspace{.1cm}{ $\begin{bmatrixcolor}[sangria]
        1 & 0& \\\\
        0 & 1  & \\\\
        0 & 0       
    \end{bmatrixcolor}}}$\\\\

    \textcolor{green}{Los vectores renglones no $0$ de $D$\\$w_{1}=(1,0)}\hspace{.5cm}{w_{2}=(0,1)}$\\\\
    \textcolor{green}{Rango(B)=2}\\

    \textcolor{green}{Espacio nulo}} \\

    $\xRightarrow{\mathit{\frac{1}{5}R_{1}}}$}\hspace{.2cm}{\hspace{.1cm}{ $\begin{bmatrixcolor}[sangria]
        1 & \frac{2}{5}& 0 & \\\\
        3 & -1 &0 & \\\\
        2 & 1  & 0       
    \end{bmatrixcolor}$\hspace{.2cm}{$\xRightarrow{\mathit{-3R_{1}+R_{2}}}}${\hspace{.1cm}{ $\begin{bmatrixcolor}[sangria]
        1 & \frac{2}{5}&0& \\\\
        0 & -\frac{11}{5}  & 0& \\\\
        2 & 1  & 0       
    \end{bmatrixcolor}}}$\hspace{.2cm}{$\xRightarrow{\mathit{2R_{1}+R_{3}}}}${\hspace{.1cm}{ $\begin{bmatrixcolor}[sangria]
        1 & \frac{2}{5}& 0 &\\\\
        0 & -\frac{11}{5}  & 0 &\\\\
        0 & -\frac{9}{5}& 0         
    \end{bmatrixcolor}}}$ \hspace{.2cm}{$\xRightarrow{\mathit{-\frac{5}{11}R_{2}}}}$\\\\
    
    $\begin{bmatrixcolor}[sangria]
        1 & \frac{2}{5}& 0 & \\\\
        0 & 1  &0 & \\\\
        0 & -\frac{9}{5}& 0         
    \end{bmatrixcolor}$\hspace{.2cm}{$\xRightarrow{\mathit{\frac{9}{5}R_{2}+R_{3}}}}${\hspace{.1cm}{ $\begin{bmatrixcolor}[sangria]
        1 & \frac{2}{5}& 0 & \\\\
        0 & 1  &0 & \\\\
        0 & 0 & 0       
    \end{bmatrixcolor}}}$\hspace{.2cm}{$\xRightarrow{\mathit{\frac{2}{5}R_{2}+R_{1}}}}${\hspace{.1cm}{ $\begin{bmatrixcolor}[sangria]
        1 & 0&0 & \\\\
        0 & 1  & 0 & \\\\
        0 & 0  &0      
    \end{bmatrixcolor}}}$\\\\

    \textcolor{green}{Sistema Equivalente \\\\ $x_{1}=0 \\\\ x_{2}=0 $\\\\ $\therefore$ el sistema no tiene solución}





    \end{enumerate}



    












\end{enumerate}





\end{document}