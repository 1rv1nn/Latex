\documentclass[a4paper,10pt]{article} 
\usepackage[top=2cm,bottom=2cm,left=2cm,rigth=2cm,heightrounded]{geometry}
\usepackage[utf8]{inputenc}
\usepackage{graphicx}
\usepackage{multirow} 
\usepackage[spanish]{babel}
\usepackage[usenames]{color}
\usepackage{dsfont}
\usepackage{amssymb}
\usepackage{amsmath}
\usepackage{bbding}  
\usepackage[dvipsnames]{xcolor}
\usepackage{csquotes}
\usepackage[export]{adjustbox}
\usepackage[all]{nowidow} 
\usepackage{csquotes} 
\everymath{\displaystyle}
\usepackage{setspace}
\usepackage[yyyymmdd]{datetime} 
\renewcommand{\dateseparator}{-} 
\usepackage{fancyhdr}
\usepackage{amsmath,xcolor}
\usepackage[inline]{enumitem}
\usepackage{amsmath} 
\usepackage{arydshln}
\usepackage{mathtools}
%Extras
\usepackage{wasysym} %caritas 
\usepackage{wasysym}
\usepackage{tikz}
\newcommand*\circled[1]{\tikz[baseline=(char.base)]{
            \node[shape=circle,draw,inner sep=2pt] (char) {#1};}}

\makeatletter
\newcommand{\xRightarrow}[2][]{\ext@arrow 0359\Rightarrowfill@{#1}{#2}}
\makeatother

\newenvironment{bmatrixcolor}[1][red]
  {\colorlet{savethecolor}{.}\colorlet{bracecolor}{#1}%
    \color{bracecolor}\left[\color{savethecolor}\begin{matrix}}
  {\end{matrix}\color{bracecolor}\right]}

%colorcitos
\definecolor{st.patrick\'sblue}{rgb}{0.14, 0.16, 0.48}
\definecolor{iris}{rgb}{0.35, 0.31, 0.81}
\definecolor{outrageousorange}{rgb}{1.0, 0.43, 0.29}
\definecolor{folly}{rgb}{1.0, 0.0, 0.31}
\definecolor{oceanboatblue}{rgb}{0.0, 0.47, 0.75}
\definecolor{cambridgeblue}{rgb}{0.64, 0.76, 0.68}
\definecolor{yellow-green}{rgb}{0.6, 0.8, 0.2}
\definecolor{unmellowyellow}{rgb}{1.0, 1.0, 0.4}
\pagecolor{cambridgeblue}

\definecolor{mediumspringgreen}{rgb}{0.0, 0.98, 0.6}
\definecolor{persianrose}{rgb}{1.0, 0.16, 0.64}
\definecolor{cadmiumyellow}{rgb}{1.0, 0.96, 0.0}
\definecolor{cyan(process)}{rgb}{0.0, 0.72, 0.92}
\definecolor{orange-red}{rgb}{1.0, 0.27, 0.0}
\definecolor{antiquecadmiumyellow}{rgb}{0.98, 0.92, 0.84} %Colores de matrices.
\definecolor{pistachio}{rgb}{0.58, 0.77, 0.45}

\pagestyle{fancy} 
\fancyhead{}\renewcommand{\headrulewidth}{0pt} 
\fancyfoot[C]{} 
\fancyfoot[R]{\thepage} 
\newcommand{\note}[1]{\marginpar{\scriptsize \textcolor{red}{#1}}} 
\begin{document}
\fancyhead[C]{}
\begin{minipage}{0.295\textwidth} 
\raggedright
Equipo\\    
\footnotesize 
\colorbox[rgb]{1.0, 1.0, 0.4}{\textcolor{black}{Aguilar Valenzuela Montserrat}}
\\\colorbox[rgb]{0.58, 0.77, 0.45}{Cruz González Irvin Javier}
\\\colorbox[rgb]{0.99, 0.76, 0.8}{\textcolor{black}{Murrillo Rosas Estefania}}
\textcolor[rgb]{1.0, 0.0, 0.31}{\medskip\hrule}
\end{minipage}
\begin{minipage}{0.4\textwidth} 
\centering 
\large 
\textbf{Matemáticas para las Ciencias Aplicadas II}\\ 
\normalsize 
Tarea 7\\
\end{minipage}
\begin{minipage}{0.295\textwidth} 
\raggedleft
\today\\ 
\footnotesize
mnts94@ciencias.unam.mx
1rv1n@ciencias.unam.mx
hollymol7@ciencias.unam.mx 
\textcolor[rgb]{1.0, 0.0, 0.31}{\medskip\hrule}
\end{minipage}

\begin{enumerate}
    \item Para cada operador lineal $T : V \rightarrow V$ , si es posible, encuentra una base $B$ para $V$ tal que la matriz relativa
    asociada a $T$ en términos de $B$ es diagonal.

        \begin{enumerate}
            \item $V=\mathds{R}^{3}$y $T(x,y,z)=(4x+z,2x+3y+2z,x+4z)$
            \item $V=\mathds{R}^{2}$y $T(x,y)=(x+2y,y)$
            \item $V=\mathds{R}^{2}$y $T(x,y)=(-2x+3y,10x+9y)$
        \end{enumerate}

        {\color{cadmiumyellow} \rule{\linewidth}{0.5mm} } 
        \begin{enumerate}
          
          \item  $\begin{bmatrixcolor}[iris]
            4 & 0 & 1    \\
            2 & 3 & 2   \\
            1 & 0 & 4          
        \end{bmatrixcolor}$ \hspace{.5cm}{matriz asociada }\hspace{2cm} {Polinomio característico $P(\lambda)=det(A-\lambda)$ }

        \hspace*{8.6cm}{$\begin{bmatrixcolor}[iris]
          4-\lambda & 0 & 1    \\
          2 & 3-\lambda & 2   \\
          1 & 0 & 4-\lambda          
      \end{bmatrixcolor}$} \\\\\\
      \textcolor{folly}{ \circled{1}} \\
      $det(A-\lambda I)=3-\lambda \begin{bmatrixcolor}[iris]
        4-\lambda & 1     \\
        1 & 4-\lambda            
    \end{bmatrixcolor}$ \\ \textit{Desarrolando}\\
    $3-\lambda[(4-\lambda)^{2}-1]$\hspace{3.5cm}{De manera que el polinomio característico}\\
    $3-\lambda[x^{2}-8x+15]$\hspace{3.5cm}{se factoriza completamente cumpliendose }\\
    $3-\lambda[(\lambda-3)(\lambda-5)]$ \hspace{3.1cm}{asi la condición 1 del test de diagonización.}\\
    $(3-\lambda)^{2}(5-\lambda)=0$ \hspace{4cm}{\textcolor{persianrose}{valores propios: $\lambda_{1}=3$ ,$\lambda_{2}=5$}}\\
    $\lambda_{1}=3 \hspace{.5cm}{\lambda_{2}=5}$\hspace{.5cm}{ \Checkmark} \hspace{3.6cm}{\textcolor{persianrose}{multiplicidad respectiva $2$,$1$}}\\

    \textcolor{folly}{ \circled{2}} \\

    $[T]_{B}-\lambda_{1}I=$ $\begin{bmatrixcolor}[iris]
      1 & 0 & 1    \\
      2 & 0 & 2   \\
      1 & 0 & 1          
  \end{bmatrixcolor}$ \hspace{.2cm}{$\xRightarrow{\mathit{-2-F_{1}+F_{2}}}$}\hspace{.2cm}{$\begin{bmatrixcolor}[iris]
    1 & 0 & 1    \\
    0 & 0 & 0   \\
    1 & 0 & 1          
\end{bmatrixcolor}$}\hspace{.2cm}{$\xRightarrow{\mathit{F_{1}-F_{3}}}$}\hspace{.2cm}{$\begin{bmatrixcolor}[outrageousorange]
  1 & 0 & 1    \\
  0 & 0 & 0   \\
  0 & 0 & 0          
\end{bmatrixcolor}$}\\\\

$rank=1$ $n-rank([T]_{B}-\lambda_{1}I)=3-1=2$ \Checkmark \hspace{1cm}{multiplicidad de $\lambda_{1}$}\\

$[T]_{B}-\lambda_{2}I=$ $\begin{bmatrixcolor}[iris]
  -1 & 0 & 1    \\
  2 & -2 & 2   \\
  1 & 0 & -1          
\end{bmatrixcolor}$ \hspace{.2cm}{$\xRightarrow{\mathit{-2-F_{1}+F_{2}}}$}\hspace{.2cm}{$\begin{bmatrixcolor}[iris]
-1 & 0 & 1    \\
0 & -2 & 4   \\
1 & 0 & -1          
\end{bmatrixcolor}$}\hspace{.2cm}{$\xRightarrow{\mathit{F_{1}-F_{3}}}$}\hspace{.2cm}{$\begin{bmatrixcolor}[red]
-1 & 0 & 1    \\
0 & -2 & 4   \\
0 & 0 & 0          
\end{bmatrixcolor}$}\\\\

$rank=2$ $n-rank([T]_{B}-\lambda_{1}I)=3-2=1$ \Checkmark \hspace{1cm}{multiplicidad de $\lambda_{2}$}\\

$\therefore $ la condición 2 satisface $\lambda_{1}$ y $\lambda_{2}$ y $T$ es diagonizable. \\\\
\newmoon \hspace{.3cm}{$E\lambda_{1}=\{
  \begin{bmatrixcolor}[iris]
    x_{1}  \\
    x_{2}    \\
    x_{3}          
  \end{bmatrixcolor} \in \mathds{R}^{3}: \begin{bmatrixcolor}[outrageousorange]
    1 & 0 & 1  \\
    0 & 0 & 0    \\
    0 & 0 & 0          
  \end{bmatrixcolor} \hspace{.1cm}{\begin{bmatrixcolor}[iris]
    x_{1}  \\
    x_{2}    \\
    x_{3}          
  \end{bmatrixcolor}} \hspace{.1cm}{=\begin{bmatrixcolor}[iris]
    0  \\
    0    \\
    0         
  \end{bmatrixcolor}}$ \}\\\\

$x_{1}+x_{3}=0$\hspace{1cm}{$x_{2}=s$}\\$x_{1}=-x_{3}$\hspace{1cm}{$x_{3}=t$ libres} \\

$E\lambda_{1}=\{\begin{bmatrixcolor}[iris]
  x_{1}  \\
  x_{2}    \\
  x_{3}          
\end{bmatrixcolor} =\begin{bmatrixcolor}[iris]
  -t \\
  s    \\
  t          
\end{bmatrixcolor}\}$ $=t\begin{bmatrixcolor}[iris]
  -1 \\
  0    \\
  1          
\end{bmatrixcolor}+s\begin{bmatrixcolor}[iris]
  0 \\
  1    \\
  0          
\end{bmatrixcolor}$ \hspace{1cm}{$B_{1}=\{ \begin{bmatrixcolor}[iris]
  -1  \\
  0    \\
  1          
\end{bmatrixcolor} \begin{bmatrixcolor}[iris]
  0  \\
  1    \\
  0          
\end{bmatrixcolor}\}$ una base para $E\lambda_{1}$} 

\newpage

\newmoon \hspace{.3cm}{$E\lambda_{2}=\{
  \begin{bmatrixcolor}[iris]
    x_{1}  \\
    x_{2}    \\
    x_{3}          
  \end{bmatrixcolor} \in \mathds{R}^{3}: \begin{bmatrixcolor}[red]
    -1 & 0 & 1  \\
    0 & 1 & 2    \\
    0 & 0 & 0          
  \end{bmatrixcolor} \hspace{.1cm}{\begin{bmatrixcolor}[iris]
    x_{1}  \\
    x_{2}    \\
    x_{3}          
  \end{bmatrixcolor}} \hspace{.1cm}{=\begin{bmatrixcolor}[iris]
    0  \\
    0    \\
    0         
  \end{bmatrixcolor}}$ \}\\\\

$x_{3}=x_{1}$ \hspace{1cm}{$x_{1}=t$}\\$x_{2}=2x_{3}$\hspace{1cm}{$x_{2}=2t$ libres} \\

$E\lambda_{2}=\{\begin{bmatrixcolor}[iris]
  x_{1}  \\
  x_{2}    \\
  x_{3}          
\end{bmatrixcolor} =\begin{bmatrixcolor}[iris]
  t \\
  2t   \\
  t          
\end{bmatrixcolor}\}$ $=t\begin{bmatrixcolor}[iris]
  1 \\
  2   \\
  1          
\end{bmatrixcolor}$ una base para $\lambda_{2}$ \\

$B^{1}=B_{1} \{$\hspace{1cm}{$B_{1}=\{ \begin{bmatrixcolor}[iris]
  -1  \\
  0    \\
  1          
\end{bmatrixcolor} \begin{bmatrixcolor}[iris]
  0  \\
  1    \\
  0          
\end{bmatrixcolor}$ $\begin{bmatrixcolor}[iris]
  1 \\
  2   \\
  1          
\end{bmatrixcolor}$\}  \\\\

$T(-1,0,1)=\begin{bmatrixcolor}[iris]
  4 & 0 & 1    \\
  2 & 3 & 2   \\
  1 & 0 & 4          
\end{bmatrixcolor}\begin{bmatrixcolor}[iris]
  -1  \\
  0    \\
  1          
\end{bmatrixcolor}=\begin{bmatrixcolor}[iris]
  -3  \\
  0   \\
  3          
\end{bmatrixcolor}=3\begin{bmatrixcolor}[iris]
  -1  \\
  0    \\
  1          
\end{bmatrixcolor}+0\begin{bmatrixcolor}[iris]
  0  \\
  1    \\
  0          
\end{bmatrixcolor}+0\begin{bmatrixcolor}[iris]
  1  \\
  2    \\
  1          
\end{bmatrixcolor} $ \\\\


$T(0,1,0)=\begin{bmatrixcolor}[iris]
  4 & 0 & 1    \\
  2 & 3 & 2   \\
  1 & 0 & 4          
\end{bmatrixcolor}\begin{bmatrixcolor}[iris]
  0 \\
  1    \\
  0          
\end{bmatrixcolor}=\begin{bmatrixcolor}[iris]
  0  \\
  3    \\
  0          
\end{bmatrixcolor}=0\begin{bmatrixcolor}[iris]
  -1  \\
  0    \\
  1          
\end{bmatrixcolor}+3\begin{bmatrixcolor}[iris]
  0  \\
  1    \\
  0          
\end{bmatrixcolor}+0\begin{bmatrixcolor}[iris]
  1  \\
  2    \\
  1          
\end{bmatrixcolor} $ \\\\


$T(1,2,1)=\begin{bmatrixcolor}[iris]
  4 & 0 & 1    \\
  2 & 3 & 2   \\
  1 & 0 & 4          
\end{bmatrixcolor}\begin{bmatrixcolor}[iris]
  1  \\
  2    \\
  1          
\end{bmatrixcolor}=\begin{bmatrixcolor}[iris]
  5  \\
  10    \\
  15         
\end{bmatrixcolor}=0\begin{bmatrixcolor}[iris]
  -1  \\
  0    \\
  1          
\end{bmatrixcolor}+0\begin{bmatrixcolor}[iris]
  0  \\
  1    \\
  0          
\end{bmatrixcolor}+5\begin{bmatrixcolor}[iris]
  1  \\
  2    \\
  1          
\end{bmatrixcolor} $ \\\\

$[T]B^{'}=\begin{bmatrixcolor}[unmellowyellow]
  3 & 0& 0  \\
  0  &3 & 0\\
  0 &  0  & 5          
\end{bmatrixcolor} $ \hspace{1cm}{$P=\begin{bmatrixcolor}[green]
  -1 & 0& 1  \\
  0  &1 & 2\\
  1 &  0  & 1          
\end{bmatrixcolor} $} \\\\

\textit{Comprobación}\\

$ P^{-1}= \left[
  \begin{array}{ccc:ccc}
  -1 &0 &1 &  1 & 0 & 0\\ 
  0 &1 &2 &  0 & 1 & 0\\
  1 &0 &1 &  0 & 0 & 1\\ 
  \end{array} \right]$ \hspace{.2cm}{$\xRightarrow{\mathit{-1F_{1}}}}}$}$ \left[
    \begin{array}{ccc:ccc}
    1 &0 &-1 &  -1 & 0 & 0\\ 
    0 &1 &2 &  0 & 1 & 0\\
    1 &0 &1 &  0 & 0 & 1\\ 
    \end{array} \right]$ \hspace{.2cm}{$\xRightarrow{\mathit{-1F_{1}+F_{3}}}}}$}$ \left[
      \begin{array}{ccc:ccc}
      1 &0 &-1 &  -1 & 0 & 0\\ 
      0 &1 &2 &  0 & 1 & 0\\
      0 &0 &1 &  0 & 0 & 1\\ 
      \end{array} \right]$ \\\\
      
      $\xRightarrow{\mathit{\frac{1}{2}F_{3}}}$$ \left[
        \begin{array}{ccc:ccc}
        1 &0 &-1 &  -1 & 0 & 0\\ 
        0 &1 &2 &  0 & 1 & 0\\
        0 &0 &2 &  0 & 0 & 1\\ 
        \end{array} \right]$ \hspace{.2cm}{$\xRightarrow{\mathit{-2F_{3}+F_{2}}}}}$}$ \left[
          \begin{array}{ccc:ccc}
          1 &0 &-1 &  -1 & 0 & 0\\\\ 
          0 &1 &2 &  0 & 1 & 0\\\\
          0 &0 &1 &  \frac{1}{2} & 0 & \frac{1}{2}\\ 
          \end{array} \right]$\hspace{.2cm}{$\xRightarrow{\mathit{F_{3}+F_{1}}}}}$}$ \left[
            \begin{array}{ccc:ccc}
            1 &0 &-1 &  -1 & 0 & 0\\\\ 
            0 &1 &0 &  0 & 1 & 0\\\\
            0 &0 &1 &  \frac{1}{2} & 0 & \frac{1}{2}\\ 
            \end{array} \right]$ \\\\

            $\xRightarrow{\mathit{F_{3}+F_{1}}}}}$$ \left[
            \begin{array}{ccc:ccc}
            1 &0 &0 &  -\frac{1}{2} & 0 & \frac{1}{2}\\\\ 
            0 &1 &0 &  - 1& 1 & -1\\\\
            0 &0 &1 &  \frac{1}{2} & 0 & \frac{1}{2}\\ 
            \end{array} \right]$ \\\\

            \textit{Ahora}\\
            $P^{-1}[T]_{B}=\begin{bmatrixcolor}[cyan]
              -\frac{1}{2} & 0& -\frac{1}{2}  \\\\
              1  &1 & 1\\\\
              \frac{1}{2} &  0  & \frac{1}{2}          
            \end{bmatrixcolor}$ $\begin{bmatrixcolor}[iris]
              4 & 0 & 1    \\
              2 & 3 & 2   \\
              1 & 0 & 4          
          \end{bmatrixcolor}$ $= \begin{bmatrixcolor}[yellow]
            -\frac{3}{2} & 0& -\frac{3}{2}  \\\\
            3  &3 & -3\\\\
            \frac{5}{2} &  0  & \frac{5}{2}          
          \end{bmatrixcolor}$ \\\\

          $P^{-1}[T]_{B}P$ $= \begin{bmatrixcolor}[yellow]
            -\frac{3}{2} & 0& -\frac{3}{2}  \\\\
            3  &3 & -3\\\\
            \frac{5}{2} &  0  & \frac{5}{2}          
          \end{bmatrixcolor}$ $\begin{bmatrixcolor}[green]
            -1 & 0& 1  \\
            0  &1 & 2\\
            1 &  0  & 1          
          \end{bmatrixcolor}$ $=\begin{bmatrixcolor}[unmellowyellow]
            3 & 0& 0  \\
            0  &3 & 0\\
            0 &  0  & 5          
          \end{bmatrixcolor} =[T]$ \textcolor{red}{\Checkmark \Checkmark }\\\\


          %%%%  EJERCICIO B %%%%%

        \item 
         $\begin{bmatrixcolor}[oceanboatblue]
          1 & 2    \\
          0 & 1    \\           
      \end{bmatrixcolor}$ \hspace{.5cm}{matriz asociada }\hspace{2cm} {Polinomio característico $P(\lambda)=det(A-\lambda)$ } \\\\

      \textcolor{folly}{ \circled{1}} \\
      $det(A-\lambda I)=\begin{bmatrixcolor}[oceanboatblue]
        1-\lambda & 2     \\
        0 & 1-\lambda            
    \end{bmatrixcolor}$ \\\\

    \textit{Desarrollando}  

    $(1-\lambda)(1-\lambda)$\\ \\
    $(1-\lambda)^{2}$ \hspace{5cm}{De manera que el polinomio caracterı́stico}\\
   $\lambda_{1}=1$ \hspace{5.5cm}{se factoriza completamente cumpliendose} \\
   \hspace*{6.5cm}{asi la condición 1 del test de diagonización.} \\
    \hspace*{8cm}{\hspace{.5cm}{\textcolor{persianrose}{valor propio: $\lambda_{1}=1$}\\
    \hspace*{8.9cm}{\textcolor{persianrose}{multiplicidad $2$}   \\

    \textcolor{folly}{ \circled{2}} \\

    $[T]_{B}-\lambda_{1}I=$ $\begin{bmatrixcolor}[oceanboatblue]
      0 & 0     \\
      2 & 0    \\          
  \end{bmatrixcolor}$ \\\\

  $rank=1$ $n-rank([T]_{B}-\lambda_{1}I)=2-1=1 \neq 2$ \XSolidBrush \\\\
  \textcolor{red}{$\therefore T(x,y)=(x+2y,y)$ NO es diagonizable}\\\\


  \item $\begin{bmatrixcolor}[cadmiumyellow]
    -2 & 3    \\
    10 & 9    \\           
\end{bmatrixcolor}$ \hspace{.5cm}{matriz asociada }\hspace{2cm} {Polinomio característico $P(\lambda)=det(A-\lambda)$ } \\\\

\textcolor{folly}{ \circled{1}} \\
$det(A-\lambda I)=\begin{bmatrixcolor}[oceanboatblue]
  -2-\lambda & 3     \\
  10 & 9-\lambda            
\end{bmatrixcolor}$ \\\\

\textit{Desarrollando}\\
$(-2-\lambda)(9-\lambda)-(30)\\
-18+2\lambda-9\lambda+\lambda^{2}-30\\
\lambda^{2}-7\lambda-48=0  $\\\\
 \smiley \textit{Resolviendo con ecuación cuadratica }\\

 $\lambda_{1}=\frac{7+\sqrt{241}}{2}$\hspace{.8cm}{$\lambda_{2}=\frac{7-\sqrt{241}}{2}$}  \\

 













        \end{enumerate}









\end{enumerate}


\end{document}