\documentclass[a4paper,10pt]{article} 
\usepackage[top=2cm,bottom=2cm,left=2cm,rigth=2cm,heightrounded]{geometry}
\usepackage[utf8]{inputenc}
\usepackage{graphicx}
\usepackage{multirow} 
\usepackage[spanish]{babel}
\usepackage[usenames]{color}
\usepackage{dsfont}
\usepackage{amssymb}
\usepackage{amsmath}
\usepackage{bbding}  
\usepackage[dvipsnames]{xcolor}
\usepackage{csquotes}
\usepackage[export]{adjustbox}
\usepackage[all]{nowidow} 
\usepackage{csquotes} 
\everymath{\displaystyle}
\usepackage{setspace}
\usepackage[yyyymmdd]{datetime} 
\renewcommand{\dateseparator}{-} 
\usepackage{fancyhdr}
\usepackage{amsmath,xcolor}
\usepackage[inline]{enumitem}
\usepackage{amsmath} 
\usepackage{arydshln}
\usepackage{mathtools}
%Extras para esta tarea.
\usepackage{tikz}
\newcommand*\circled[1]{\tikz[baseline=(char.base)]{
            \node[shape=circle,draw,inner sep=2pt] (char) {#1};}}


\makeatletter
\newcommand{\xRightarrow}[2][]{\ext@arrow 0359\Rightarrowfill@{#1}{#2}}
\makeatother

\newenvironment{bmatrixcolor}[1][red]
  {\colorlet{savethecolor}{.}\colorlet{bracecolor}{#1}%
    \color{bracecolor}\left[\color{savethecolor}\begin{matrix}}
  {\end{matrix}\color{bracecolor}\right]}

%%%%%AQUÍ SE ENCUENTRAN LOS COLORES :)%%%%%%%%
\definecolor{mediumspringgreen}{rgb}{0.0, 0.98, 0.6}
\definecolor{persianrose}{rgb}{1.0, 0.16, 0.64}
\definecolor{cadmiumyellow}{rgb}{1.0, 0.96, 0.0}
\definecolor{cyan(process)}{rgb}{0.0, 0.72, 0.92}
\definecolor{orange-red}{rgb}{1.0, 0.27, 0.0}
\definecolor{antiquecadmiumyellow}{rgb}{0.98, 0.92, 0.84} %Colores de matrices.
\definecolor{apricot}{rgb}{0.98, 0.81, 0.69}
\definecolor{babypink}{rgb}{0.96, 0.76, 0.76}
\definecolor{guppiegreen}{rgb}{0.0, 1.0, 0.5}
\definecolor{maize}{rgb}{0.98, 0.93, 0.37}
\definecolor{redwood}{rgb}{0.67, 0.31, 0.32}
\definecolor{rosybrown}{rgb}{0.74, 0.56, 0.56}
\definecolor{amber}{rgb}{1.0, 0.75, 0.0}
\definecolor{sangria}{rgb}{0.57, 0.0, 0.04}



\definecolor{burgundy}{rgb}{0.5, 0.0, 0.13} %---> Este color me gusta para la proxima tarea:)
\definecolor{bulgarianrose}{rgb}{0.28, 0.02, 0.03}
\definecolor{bananamania}{rgb}{0.98, 0.91, 0.71} %---> Este color me gusta para matrices :)
\pagecolor{black}
\color{white}



\pagestyle{fancy} 
\fancyhead{}\renewcommand{\headrulewidth}{0pt} 
\fancyfoot[C]{} 
\fancyfoot[R]{\thepage} 
\newcommand{\note}[1]{\marginpar{\scriptsize \textcolor{red}{#1}}} 
\begin{document}
\fancyhead[C]{}
\begin{minipage}{0.295\textwidth} 
\raggedright
Equipo\\    
\footnotesize 
\colorbox[rgb]{0.96, 0.73, 1.0}{\textcolor{black}{Aguilar Valenzuela Montserrat}}
\\\colorbox[rgb]{0.98, 0.93, 0.36}{\textcolor{black}{Cruz González Irvin Javier}}
\\\colorbox[rgb]{0.94, 1.0, 0.94}{\textcolor{black}{Murrillo Rosas Estefania}}
\textcolor[rgb]{0.74, 0.83, 0.9}{\medskip\hrule}
\end{minipage}
\begin{minipage}{0.4\textwidth} 
\centering 
\large 
\textbf{Matemáticas para las Ciencias Aplicadas II}\\ 
\normalsize 
Tarea 6\\
\end{minipage}
\begin{minipage}{0.295\textwidth} 
\raggedleft
\today\\ 
\footnotesize
mnts94@ciencias.unam.mx
1rv1n@ciencias.unam.mx
hollymol7@ciencias.unam.mx 
\textcolor[rgb]{0.74, 0.83, 0.9}{\medskip\hrule}
\end{minipage}

\begin{enumerate}
  
\item Demuestra si las siguientes funciones son transformaciones lineales.
      \begin{enumerate}
        \item $T:R^3\rightarrow R^3,T(x,y,z)=(x+y,x-y,z)$
        \item $T:R^3\rightarrow R^3,T(x,y,z)=(x^2,xy,y^2)$
      \end{enumerate}

      {\color{amber} \rule{\linewidth}{0.5mm} }      

      \begin{enumerate}
        
        \item \textcolor{guppiegreen}{ \circled{1}} \hspace{.2cm} {\textcolor{guppiegreen}{$T(\vec{u}+\vec{v})=T(\vec{u})+T(\vec{v})$}} \\\\
              Sea $\vec{x},\vec{y},\vec{z} \in \mathds{R}^{3}$ con $\vec{x}(x_{1},x_{2},x_{3})$ $\vec{y}(y_{1},y_{2},y_{3})$ $\vec{z}(z_{1}+z_{2}+z_{3})$ y $c \in \mathds{R}$\\\\
              Entonces $\vec{x}+\vec{y}+\vec{z}(x_{1}+y_{1}+z_{1},x_{2}+y_{2}+z_{2},x_{3}+y_{3}+z_{3})$\\\\
              $T(\vec{u}+\vec{v}) =$ $ T((v_{1},v_{2},v_{3})+(v_{1},v_{2},v_{3}))$ \hspace{3.8cm}{\textcolor{cyan}{\textit{Sustituyendo vectores}}}\\\\
              \hspace*{1.4cm}{$=T(u_{1}+v_{1},u_{2}+v_{2},u_{3}+v_{3})$}\hspace{3.9cm}{\textcolor{cyan}{\textit{Definición de suma}}}\\\\
              \hspace*{1.4cm}{$=(u_{1}+v_{1},u_{2}+v_{2},u_{3}+v_{3})$}\hspace{4.1cm}{\textcolor{cyan}{\textit{Aplicando transformación}}}\\\\
              \hspace*{1.4cm}{$=(u_{1}+v_{1}+u_{2}+v_{2},u_{1}+v_{1}-u_{2}-v_{2},u_{3}+v_{3})$}\hspace{1cm}{\textcolor{cyan}{\textit{Algebra vectorial}}}\\\\
              \hspace*{1.4cm}{$=(u_{1}+u_{2},u_{1}-u_{2},u_{3})+(v_{1}+v_{2}+v_{1}-v_{2},v_{3})$}\hspace{1cm}{\textcolor{cyan}{\textit{Reorganizando y llevarlo}}}\\
              \hspace*{9.7cm}{\textcolor{cyan}{\textit{de la forma $(x+y,x-y,z)$}}}\\\\
              \hspace*{1.4cm}{$=x+y,x-y,z$}\\\\
              \hspace*{1.4cm}{$T(\vec{u}+T(\vec{v})) \therefore $ Se cumple \hspace{4.3cm}{\textcolor{guppiegreen}{$T(\vec{u}+\vec{v})=T(\vec{u})+T(\vec{v})$}}}\\

          
       \textcolor{guppiegreen}{\circled{2} \hspace{.2cm}{$T(c\vec{u})=cT(\vec{u})$}} \\\\
        $c(\vec{u})=C(u_{1},u_{2},u_{3})=(Cu_{1},Cu_{2},Cu_{3})$\\\\
        $T(C\vec{u})=T(Cu_{1},Cu_{2},Cu_{3})=(Cu_{1}+Cu_{2},Cu_{1}-Cu_{2},Cu_{3})$ \hspace{.7cm}{\textcolor{cyan}{\textit{aplic transformación}}}\\\\
        \hspace*{1.4cm}{$=C(u_{1}+u_{2},u_{1}-u_{2},u_{3})$}\\\\
        \hspace*{1.4cm}{\textcolor{guppiegreen}{$=cT(\vec{u})$}} \\\\
        \textcolor{persianrose}{$\therefore $ $T$ es una transformación lineal. }


        \item \textcolor{persianrose}{No es una transformación lineal de $\mathds{R}^3$ a $\mathds{R}^{3}$ ya que en general $(x_{1}+x_{2})^{2} \neq (x_{1})^{2} + (x_{2})^{2}$ }


      \end{enumerate}



      
\item En lo siguientes incisos considera la transformación lineal $T:R^n \rightarrow R^m$como $T(v)=Av$
      \begin{enumerate}
        \item $\begin{bmatrixcolor}[bananamania]
          1 & 2    \\
          -2 & 4   \\
          -2 & 2          
      \end{bmatrixcolor}$\\
              \begin{enumerate}
                \item [i)] Determina las dimensiones de $R^n$ y $R^m$.
                \item [ii)] Encuentra $T(2, 4)$.
                \item [iii)] Calcula la preimagen de $(-1, 2, 2)$.
                \item [iv)] Explica porque el vector $(1, 1, 1)$ no tiene preimagen bajo esta transformación.\\
              \end{enumerate}                           

        \item  $\begin{bmatrixcolor}[mediumspringgreen]
          0 & 2 & 0 & 2 & 0    \\
          1 & 0 & 1 & 0 & 1    \\
          1 & 2 & 2 & 2 & 1            
      \end{bmatrixcolor}$ \\
               \begin{enumerate}
                 \item [i)] Determina las dimensiones de $R^n$ y $R^m$.
                 \item [ii)] Encuentra $T(0, 1, 0, 1, 0)$.
                 \item [iii)] Calcula la preimagen de $(0, 0, 0)$.
                 \item [iv)] Calcula la preimagen de $(1, -1, 2)$.
               \end{enumerate}    
      
      \end{enumerate} 

      {\color{amber} \rule{\linewidth}{0.5mm} }  

      \begin{enumerate}
        \item 
        \begin{enumerate}
          \item [i)] \textcolor{persianrose}{El tamaño de la matriz es de $3$x$2$ asi que la transformación lineal esta definida desde $\mathds{R}^2$ a $\mathds{R}^3$.}\\
          
          $A\textbf{v}=$ $\begin{bmatrixcolor}[bananamania]
            1 & 2    \\
            -2 & 4   \\
            -2 & 2          
        \end{bmatrixcolor}$\hspace{.1cm}{ $\begin{bmatrixcolor}[bananamania]
          v_{1}     \\
          v_{2}         
        \end{bmatrixcolor}$}\hspace{.1cm}{ $=\begin{bmatrixcolor}[bananamania]
          u_{1}     \\
          u_{2}     \\
          u_{3}    
        \end{bmatrixcolor}$} \textcolor{cyan}{$\longleftarrow$}\textcolor{cyan}{\framebox[1.1\width]{\textcolor{white}{vector en $\mathds{R}^3$}}}\\

          \hspace*{2.9cm}\textcolor{cyan}{{$\big\uparrow$}} \\       
          \hspace*{2cm}\textcolor{cyan}{\framebox[1.1\width]{\textcolor{white}{vector en $\mathds{R}^2$}}}\\\\

        \item [ii)]
        $\begin{bmatrixcolor}[bananamania]
          1 & 2    \\
          -2 & 4   \\
          -2 & 2          
      \end{bmatrixcolor}$\hspace{.1cm}{ $\begin{bmatrixcolor}[bananamania]
        2     \\
        4         
      \end{bmatrixcolor}$}\hspace{.1cm}{ $=\begin{bmatrixcolor}[bananamania]
        10     \\
        12     \\
        4    
      \end{bmatrixcolor}$} \hspace{1cm}{\textcolor{persianrose}{$\therefore T(2,4)=(10,12,4)$}}\\\\



        \item [iii)] $\vec{v},T(\vec{v})=(-1,2,2)$\\\\
                     $\vec{v}=(v_{1},v_{2}) \in \mathds{R}^2=Dom(T)$ \\\\
                     $T(v_{1},v_{2},v_{3})=(v_{1}+2v_{2},-2v_{1}+4v_{2},-2v_{1}+2v_{2})$\\\\
                     $v_{1}+2v_{2}=-1$\hspace{.7cm}{\textcolor{yellow}{....1}} \hspace{3.5cm}{De \textcolor{yellow}{ 1}}\\
                     $-2v_{1}+4v_{2}=2$\hspace{.5cm}{\textcolor{yellow}{....2}}\hspace{3.5cm}{$v_{1}=-1-2v_{2}$} \hspace{1cm}{$v_{1}=-1$}\\ 
                     $-2v_{1}+2v_{2}=2$\hspace{.5cm}{\textcolor{yellow}{....3}}\hspace{3.7cm}{De \textcolor{yellow}{ 3}}\\
                     \hspace*{7cm}{$2v_{2}=2+2v_{1}$}\\
                     \hspace*{7cm}{$2v_{2}=2+2(-1-2v_{2})$}\\
                     \textcolor{persianrose}{$\therefore T^{-1}(-1,2,2)=(-1,0)$}\hspace{3.1cm}{$2v_{2}=2-2-4v_{2}$}\\
                     \hspace*{7cm}{$4v_{2}+2v_{2}=0$}\\
                     \hspace*{7cm}{$6v_{2}=0$}\\\\
                     \hspace*{7cm}{$v_{2}=\frac{0}{6}$}\hspace{.6cm}{$v_{2}=0$}




        \item [iv)] \textcolor{persianrose}{Dado que al resolver el sistema de ecuaciones el resultado es inconsistente con $0 \neq \frac{3}{4}$ no puede mostrar que $T^{-1}(1,1,1)$ pertenece a $\mathds{R}^{2}$.}\\
        
        $\vec{v},T(\vec{v})=(1,1,1)$\\\\
        $\vec{v}=(v_{1},v_{2}) \in \mathds{R}^2=Dom(T)$ \\\\
        $T(v_{1},v_{2},v_{3})=(v_{1}+2v_{2},-2v_{1}+4v_{2},-2v_{1}+2v_{2})$\\\\
        \left\lbrace
        \begin{array}{ll}
        \textup{$v_{1}+2v_{2}=1$ } \\\\
        \textup{$-2v_{1}+4v_{2}=1$}\\\\
        \textup{$-2v_{1}+2v_{2}=1$}
        \end{array}\hspace{.2cm}{$\xRightarrow$}\hspace{.2cm}{ $\begin{bmatrixcolor}[cyan]
            1 & 2 & 1   \\
            -2 & 4 & 1 \\
            -2 & 2 & 1 
        \end{bmatrixcolor}$} \\\\

        $\xRightarrow{\mathit{2F_{1}+F_{2}}}$\hspace{.2cm}{$\begin{bmatrixcolor}[cyan]
          1 & 2 & 1   \\
          0 & 8 & 3 \\
          -2 & 2 & 1 
      \end{bmatrixcolor}$}\hspace{.2cm}{$\xRightarrow{\mathit{2F_{1}+F_{3}}}$}\hspace{.2cm}{$\begin{bmatrixcolor}[cyan]
        1 & 2 & 1   \\
        0 & 8 & 3 \\
        0 & 6 & 3 
    \end{bmatrixcolor}$}\hspace{.2cm}{$\xRightarrow{\mathit{\frac{1}{8}F_{2}}}}$}\hspace{.2cm}{$\begin{bmatrixcolor}[cyan]
      1 & 2 & 1   \\\\
      0 & 1 & \frac{3}{8} \\\\
      0 & 6 & 3 
  \end{bmatrixcolor}$} \hspace{.2cm}{$\xRightarrow{\mathit{-2F_{2}+F_{1}}}}$}\hspace{.2cm}{$\begin{bmatrixcolor}[cyan]
    1 & 0 & \frac{1}{4}   \\\\
    0 & 1 & \frac{3}{8} \\\\
    0 & 6 & 3 
\end{bmatrixcolor}$} \newpage \hspace{.2cm}{$\xRightarrow{\mathit{-6F_{2}+F_{3}}}}$}\hspace{.2cm}{$\begin{bmatrixcolor}[cyan]
  1 & 0 & \frac{1}{4}   \\\\
  0 & 1 & \frac{3}{8} \\\\
  0 & 0 & \frac{3}{4} 
\end{bmatrixcolor}$} \hspace{1cm}{\textcolor{persianrose}{El sistema es inconsistente $0\neq \frac{3}{4}$}}
      
        
        





        \end{enumerate}

      \item \begin{enumerate}
        \item [i)] \textcolor{persianrose}{El tamaño de la matriz es de $3$x$5$ asi que la transformación lineal esta definida desde $\mathds{R}^5$ a $\mathds{R}^3$.}\\
          
        $A\textbf{v}=$ $\begin{bmatrixcolor}[mediumspringgreen]
          0 & 2 & 0 & 2 & 0    \\
          1 & 0 & 1 & 0 & 1    \\
          1 & 2 & 2 & 2 & 1            
      \end{bmatrixcolor}$ \hspace{.1cm}{ $\begin{bmatrixcolor}[mediumspringgreen]
        v_{1}     \\
        v_{2} \\
        v_{3} \\
        v_{4} \\
        v_{5}         
    \end{bmatrixcolor}$}\hspace{.1cm}{ $=\begin{bmatrixcolor}[mediumspringgreen]
        u_{1}     \\
        u_{2}     \\
        u_{3}    
  \end{bmatrixcolor}$} \textcolor{cyan}{$\longleftarrow$}\textcolor{cyan}{\framebox[1.1\width]{\textcolor{white}{vector en $\mathds{R}^3$}}}\\

  \hspace*{4.3cm}\textcolor{cyan}{{$\big\uparrow$}} \\       
  \hspace*{3.3cm}\textcolor{cyan}{\framebox[1.1\width]{\textcolor{white}{vector en $\mathds{R}^5$}}}\\\\

        \item [ii)] $\begin{bmatrixcolor}[mediumspringgreen]
          0 & 2 & 0 & 2 & 0    \\
          1 & 0 & 1 & 0 & 1    \\
          1 & 2 & 2 & 2 & 1            
      \end{bmatrixcolor}$ \hspace{.1cm}{ $\begin{bmatrixcolor}[mediumspringgreen]
         0    \\
         1\\
         0\\
         1\\
          0       
    \end{bmatrixcolor}$}\hspace{.1cm}{ $=\begin{bmatrixcolor}[mediumspringgreen]
        4     \\
        0     \\
        4    
  \end{bmatrixcolor}$}  \hspace{1cm}{\textcolor{persianrose}{$\therefore T(0,1,0,1,0)=(4,0,4)$}}


        



        \item [iii)] $\vec{v},T(\vec{v})=(0,0,0)$\\\\
        $\vec{v}=(v_{1},v_{2},v_{3},v_{4},v_{5}) \in \mathds{R}^5=Dom(T)$ \\\\
        $T(v_{1},v_{2},v_{3})=(2v_{2}+2v_{4},v_{1}+v_{3}+v_{5},v_{1}+2v_{2}+2v_{3}+2v_{4}+v_{5})$\\\\
        \left\lbrace
            \begin{array}{ll}
            \textup{$2v_{2}+2v_{4}=0$ } \\\\
            \textup{$v_{1}+v_{3}+v_{5}=0$}\\\\
            \textup{$v_{1}+2v_{2}+2v_{3}+2v_{4}+v_{5}=0$}
            \end{array}\hspace{.2cm}{$\xRightarrow$}\hspace{.2cm}{ $\begin{bmatrixcolor}[cyan]
                0 & 2 & 0 & 2 & 0 & 0  \\
                1 & 0 & 1 & 0 & 1 & 0\\
                1 & 2 & 2 & 2 & 1 & 0 
            \end{bmatrixcolor}$}\\\\ \textit{\textcolor{pink}{Resolviendo con Gauss-Jordan}}}\\

            $\begin{bmatrixcolor}[cyan]
              0 & 2 & 0 & 2 & 0 & 0  \\
              1 & 0 & 1 & 0 & 1 & 0\\
              1 & 2 & 2 & 2 & 1 & 0 
          \end{bmatrixcolor}$} \hspace{.2cm}{$\xRightarrow{\mathit{F_{2}+F_{1}}}$}\hspace{.2cm}{ $\begin{bmatrixcolor}[cyan]
            1 & 2 & 1 & 2 & 1 & 0  \\
            1 & 0 & 1 & 0 & 1 & 0\\
            1 & 2 & 2 & 2 & 1 & 0 
        \end{bmatrixcolor}$} }\hspace{.2cm}{$\xRightarrow{\mathit{-F_{3}+F_{2}}}$}\hspace{.2cm}{ $\begin{bmatrixcolor}[cyan]
          1 & 2 & 1 & 2 & 1 & 0  \\
          0 & -2 & -1 & -2 & 0 & 0\\
          1 & 2 & 2 & 2 & 1 & 0 
      \end{bmatrixcolor}$} }\\\\

      $\xRightarrow{\mathit{-F_{1}+F_{3}}}$\hspace{.2cm}{ $\begin{bmatrixcolor}[cyan]
        1 & 2 & 1 & 2 & 1 & 0  \\
        0 & -2 & -1 & -2 & 0 & 0\\
        0 & 0 & 1 & 0 & 0 & 0 
    \end{bmatrixcolor}$} }\hspace{.2cm}{$\xRightarrow{\mathit{-\frac{1}{2}F_{2}}}}$}\hspace{.2cm}{ $\begin{bmatrixcolor}[cyan]
      1 & 2 & 1 & 2 & 1 & 0  \\\\
      0 & 1 & \frac{1}{2} & 1 & 0 & 0\\\\
      1 & 2 & 2 & 2 & 1 & 0 
  \end{bmatrixcolor}$} }\hspace{.2cm}{$\xRightarrow{\mathit{-2F_{2}+F_{1}}}}$}\\\\

  $\begin{bmatrixcolor}[cyan]
    1 & 0 & 0 & 0 & 1 & 0  \\\\
    0 & 1 & \frac{1}{2} & 1 & 0 & 0\\\\
    1 & 2 & 2 & 2 & 1 & 0 
\end{bmatrixcolor}$ \hspace{.2cm}{$\xRightarrow{\mathit{-\frac{1}{2}F_{3}+F_{2}}}}$}\hspace{.2cm}{$\begin{bmatrixcolor}[cyan]
  1 & 0 & 0 & 0 & 1 & 0  \\
  0 & 1 & 0 & 1 & 0 & 0\\
  1 & 2 & 2 & 2 & 1 & 0 
\end{bmatrixcolor}$} \\\\ $v_{1}+v_{5}=0$ \hspace{1cm}{Sea $T=v_{4}$,$S=v_{5}$ con $S$ y $T \in \mathds{R}$\\ $v_{2}+v_{4}=0$ \hspace{1cm}{$v_{1}=-S$} \\ $v_{3}=0 \hspace{1.8cm}{v_{2}=-T}$\\
\hspace*{2.8cm}{$v_{3}=0$} \hspace{3cm}{\textcolor{persianrose}{$\therefore T^{-1}(0,0,0)=(-S,-T,0,T,S)$} \\\\}


        \item [iv)]  $\vec{v},T(\vec{v})=(1,-1,2)$\\\\
        $\vec{v}=(v_{1},v_{2},v_{3},v_{4},v_{5}) \in \mathds{R}^5=Dom(T)$ \\\\
        $T(v_{1},v_{2},v_{3})=(2v_{2}+2v_{4},v_{1}+v_{3}+v_{5},v_{1}+2v_{2}+2v_{3}+2v_{4}+v_{5})$\\\\
        \left\lbrace
            \begin{array}{ll}
            \textup{$2v_{2}+2v_{4}=1$ } \\\\
            \textup{$v_{1}+v_{3}+v_{5}=-1$}\\\\
            \textup{$v_{1}+2v_{2}+2v_{3}+2v_{4}+v_{5}=2$}
            \end{array}\hspace{.2cm}{$\xRightarrow$}\hspace{.2cm}{ $\begin{bmatrixcolor}[cyan]
                0 & 2 & 0 & 2 & 0 & 1  \\
                1 & 0 & 1 & 0 & 1 & -1\\
                1 & 2 & 2 & 2 & 1 & 2 
            \end{bmatrixcolor}$}\\\\

            \textit{\textcolor{pink}{Resolviendo con Gauss-Jordan}}}\\

           $\xRightarrow{\mathit{F_{2}\rightleftarrows F_{1}}}$\hspace{.2cm}{ $\begin{bmatrixcolor}[cyan]
              1 & 0 & 1 & 0 & 1 & -1  \\
              0 & 2 & 0 & 2 & 0 & 1\\
              1 & 0 & 1 & 0 & 0 & 2 
          \end{bmatrixcolor}$} } \hspace{.2cm}{$\xRightarrow{\mathit{-F_{1}+F_{3}}}}$}\hspace{.2cm}{ $\begin{bmatrixcolor}[cyan]
            1 & 0 & 1 & 0 & 1 & -1  \\
            0 & 2 & 0 & 2 & 0 & 1\\
            0 & 2 & 1 & 2 & 0 & 3 
        \end{bmatrixcolor}$} } \hspace{.2cm}{$\xRightarrow{\mathit{\frac{1}{2}F_{2}}}}$} \\\\
        
        $\begin{bmatrixcolor}[cyan]
          1 & 0 & 1 & 0 & 1 & -1  \\\\
          0 & 1 & 0 & 1 & 0 & \frac{1}{2}\\\\
          0 & 2 & 1 & 2 & 0 & 3 
      \end{bmatrixcolor}$ \hspace{.2cm}{$\xRightarrow{\mathit{-2F_{2}+F_{3}}}}$} \hspace{.2cm}{ $\begin{bmatrixcolor}[cyan]
        1 & 0 & 1 & 0 & 1 & -1  \\\\
        0 & 1 & 0 & 1 & 0 & \frac{1}{2}\\\\
        0 & 0 & 1 & 0 & 0 & 2 
    \end{bmatrixcolor}$}\hspace{.2cm}{$\xRightarrow{\mathit{-F_{3}+F_{1}}}}$} $\begin{bmatrixcolor}[cyan]
      1 & 0 & 0 & 0 & 1 & -3  \\\\
      0 & 1 & 0 & 1 & 0 & \frac{1}{2}\\\\
      0 & 0 & 1 & 0 & 0 & 2 
  \end{bmatrixcolor}$ \\\\

  $v_{1}+v_{5}=-3$\hspace{1cm}{Sea $T=v_{4}$,$S=v_{5}$ con $S$ y $T \in \mathds{R}$\\
  $v_{2}+v_{4}=\frac{1}{2}$\hspace{1.2cm}{$v_{1}=-3-S$}\\
  $v_{3}=2$\hspace{2cm}{$v_{2}=\frac{1}{2}-T$} \hspace{2cm}{\textcolor{persianrose}{$\therefore T^{-1}(1,-1,2)=(-3-S,\frac{1}{2}-T,2,-S,-T)$}}\\\\
  \hspace*{3cm}{$v_{3}=2$}

      \end{enumerate} 

      \end{enumerate}


\item Para las transformaciones lineales definidas por $T(x) = Ax$, encuentra i) \textit{kernel}$(T)$, ii) \textit{nulidad}$(T)$, iii) \textit{imagen}$(T)$
y iv) \textit{rango}$(T)$.

      \begin{enumerate*}
        \item $A=\begin{bmatrixcolor}[maize]
          5 & -3    \\
          1 & 1   \\
          1 & -1          
      \end{bmatrixcolor}$ \hspace{3cm}{\hspace*{2cm}{\item $B=\begin{bmatrixcolor}[babypink]
        1 & 0 & 1    \\
        0 & 1 & 0 \\
        1 & 0 &  1       
    \end{bmatrixcolor}}}}$

      \end{enumerate*}


      {\color{amber} \rule{\linewidth}{0.5mm} }  

\begin{enumerate}
  \item \begin{enumerate}
        \item [i)] \textit{kernel} \\ El kernel es el conjuntos de todos los $\vec{x}(x_{1},x_{2})=(0,0)$ \hspace{1cm}{\textcolor{yellow}{$\mathds{R}^{2}\rightarrow \mathds{R}^{3}$}}\\
    
        $\begin{bmatrixcolor}[maize]
          5 & -3     \\
          1 & 1  \\
          1 & -1        
      \end{bmatrixcolor}$\hspace{.1cm}{$\begin{bmatrixcolor}[maize]
        x_{1} \\
        x_{2}        
    \end{bmatrixcolor}$}\hspace{.1cm}{$=\begin{bmatrixcolor}[maize]
      0 \\
      0        
  \end{bmatrixcolor}$} \hspace{.2cm}{$\rightarrow$} \hspace{.2cm}{$\begin{bmatrixcolor}[maize]
    5 & -3  & 0\\
    1 & 1  & 0\\
    1 & -1 & 0         
\end{bmatrixcolor}$} \\\\

$\xRightarrow{\mathit{F_{2}\rightleftarrows F_{1}}}}$ \hspace{.2cm}{$\begin{bmatrixcolor}[maize]
  1 & 1  & 0\\
  5 & -3  & 0\\
  1 & -1 & 0         
\end{bmatrixcolor}$}\hspace{.2cm}{$\xRightarrow{\mathit{-5F_{1} + F_{2}}}}$}\hspace{.2cm}{$\begin{bmatrixcolor}[maize]
  1 & 1  & 0\\
  0 & -8  & 0\\
  1 & -1 & 0         
\end{bmatrixcolor}$}\hspace{.2cm}{$\xRightarrow{\mathit{-F_{1} + F_{3}}}}$}\hspace{.2cm}{$\begin{bmatrixcolor}[maize]
  1 & 1  & 0\\
  0 & -8  & 0\\
  0 & -2 & 0         
\end{bmatrixcolor}$}\hspace{.2cm}{$\xRightarrow{\mathit{\frac{1}{8}F_{2}}}}$}\hspace{.2cm}{$\begin{bmatrixcolor}[maize]
  1 & 1  & 0\\
  0 & 1  & 0\\
  0 & -2 & 0         
\end{bmatrixcolor}$}\\\\

$\xRightarrow{\mathit{2F_{2}+F_{3}}}}$\hspace{.2cm}{$\begin{bmatrixcolor}[maize]
  1 & 1  & 0\\
  0 & 1  & 0\\
  0 & 0 & 0         
\end{bmatrixcolor}$}\hspace{$\xRightarrow{\mathit{-F_{2}+F_{1}}}}$}\hspace{$\begin{bmatrixcolor}[maize]
  1 & 0  & 0\\
  0 & 1  & 0\\
  0 & 0 & 0         
\end{bmatrixcolor}$}\hspace{1cm}{$x_{1}=0$}\\
\hspace*{7cm}{$x_{2}=0$} \hspace{.1cm}{$\begin{bmatrixcolor}[maize]
  x_{1} \\
  x_{2}        
\end{bmatrixcolor}$}\hspace{.1cm}{$=\begin{bmatrixcolor}[maize]
  0 \\
  0        
\end{bmatrixcolor}=$} $Ker(T)$\\\\

\textcolor{persianrose}{$\therefore Ker(T)=\{(0,0)\}$}\\
   
  \item [ii)]\textit{nulidad} \\$nulidad=dim(dominio)-rango=\textcolor{yellow}{2}\textcolor{green}{-2}=0$ \hspace{1cm}{\textcolor{persianrose}{$\therefore nulidad $ $T(x)=0$}}\\\\
  

  %imagen 1
  \item [iii)]\textit{imagen}\\\\
  $rank(T)+nulidad=dim(dominio)$\\
 
  $rank(T)=dim(dominio)-nulidad$\\\\
  $rank(T)=\textcolor{yellow}{2}-0=2$ \hspace{4cm}\textcolor{persianrose}{$\therefore imagen=\mathds{R}^2$,corresponde a un plano.}\\
 
 
  
  \item [iv)] \textit{rango}\\ número de filas de $A$ que no son nulas = \textcolor{green}{2} \hspace{1.4cm}{\textcolor{persianrose}{$\therefore$ el rango de $T(x)=2$}}\\  
        \end{enumerate}  





  \item \begin{enumerate}
        \item [i)] \textit{kernel} \\ El kernel es el conjunto de todos los $\vec{x}(x_{1},x_{2},x_{3})=(0,0)$ \hspace{1cm}{\textcolor{green}{$\mathds{R}^{3}\rightarrow \mathds{R}^{3}$}}\\
        
        $\begin{bmatrixcolor}[babypink]
          1 & 0 & 1    \\
          0 & 1 & 0 \\
          1 & 0 &  1       
      \end{bmatrixcolor}$\hspace{.1cm}{$\begin{bmatrixcolor}[babypink]
        x_{1} \\
        x_{2} \\
        x_{3}        
    \end{bmatrixcolor}$}\hspace{.1cm}{$=\begin{bmatrixcolor}[babypink]
      0 \\
      0 \\
      0       
  \end{bmatrixcolor}$} \hspace{.3cm}{$\rightarrow$}\hspace{.3cm}{  $\begin{bmatrixcolor}[babypink]
    1 & 0 & 1 & 0   \\
    0 & 1 & 0 & 0\\
    1 & 0 &  1 & 0      
\end{bmatrixcolor}$}\hspace{.2cm}{$\xRightarrow{\mathit{-F_{1}+F_{3}}}}$}\hspace{.2cm}{  $\begin{bmatrixcolor}[babypink]
  1 & 0 & 1 & 0   \\
  0 & 1 & 0 & 0\\
  0 & 0 &  0 & 0      
\end{bmatrixcolor}$} \\\\

$x_{3}=T \in \mathds{R}$ \\

$x_{1}=-T$\\ $x_{2}=0$ \hspace{1cm}{$\begin{bmatrixcolor}[babypink]
  x_{1} \\
  x_{2} \\
  x_{3}        
\end{bmatrixcolor}$}\hspace{.2cm}{$=\begin{bmatrixcolor}[babypink]
  -T \\
  0 \\
  T        
\end{bmatrixcolor}$} \hspace{3.3cm}{\textcolor{persianrose}{$\therefore Kernel(T)=\{(-T,0,T)\}  : T \in \mathds{R}$}} \\\\

\item [ii)] \textit{nulidad}\\ $nulidad=dim(dominio)-rango=\textcolor{green}{3}\textcolor{yellow}{-2}=1$ \hspace{1cm}{\textcolor{persianrose}{$\therefore nulidad$ $T(x)= 1$}}\\\\


%imagen 2
 \item [iii)] \textit{imagen}\\\\
 $rank(T)+nulidad=dim(dominio)$\\

 $rank(T)=dim(dominio)-nulidad$\\\\
 $rank(T)=\textcolor{green}{3}-1=2$ \hspace{4cm}\textcolor{persianrose}{$\therefore imagen=\mathds{R}^2$,corresponde a un plano.}\\



 \item [iv)]  \textit{rango}\\número de filas de $B$ que no son nulas = \textcolor{yellow}{2} \hspace{1.3cm}{\textcolor{persianrose}{$\therefore$ el rango de $T(x)=2$}}\\     

  \end{enumerate}      


\end{enumerate}

\newpage


\item Sea $T : R^m \rightarrow R^n$ una transformación lineal. a) Explica la diferencia entre los conceptos uno-a-uno y sobre.
¿Qué puedes decir acerca de $m$ y $n$ cuando $T$ es: b) sobre y c) uno-a-uno?    


      {\color{amber} \rule{\linewidth}{0.5mm} }   


      \item [a)]\textcolor{persianrose}{\textcolor{yellow}{Uno-uno}\\ \textcolor{cyan}{1}.El kernel de $T(x)$ solo consta del $ \{\vec{0}}\}$.\\ \textcolor{cyan}{2}.Cuales quiera 2 elementos $(\vec{u},\vec{v})$ en el codomio $(W)$,estos viene del mismo
      elemento en el dominio $(V)$}\\
                \textcolor{persianrose}{\textcolor{yellow}{Sobre}\\\textcolor{cyan}{1.} Una transformación es sobre cuando $W$ es igual a la imagen de $T$.} 


      \item [b)] \textcolor{persianrose}{Teniendo una $T(x)=Ax$.Se dice que es sobre cuando la dimensión del $rango(T)$ es igual a la dimension del codominio(n). }
      \item [c)] \textcolor{persianrose}{Teniendo una $T(x)=Ax$.Se dice que es uno a uno cuando la nulidad de $T$($domin-rango$) es cero}


\item Para las siguientes transformaciones: 
      \begin{enumerate}
        \item [i)] Encuentra la matriz estándar $A$ para la transformación lineal $T$.
        \item [ii)] Usa $A$ para encontrar la imagen del vector $v$.
        \item [iii)] Grafica el vector $v$ y su imagen.
      \end{enumerate}

      \begin{enumerate}
        \item $T(x, y)$ = $(x-3y, 2x + y, y), v = (-2, 4)$
        \item $T(x, y)$ = $(y, x),$ $v = (3, 4)$
      \end{enumerate}


      {\color{amber} \rule{\linewidth}{0.5mm} }

      \begin{enumerate}
        \item \begin{enumerate}
              \item [i)] $T(\vec{e_{1}}) = T  (1,0,0)=(1,2,0)$ \\
                         $T(\vec{e_{2}})=T(0,1,0) = (-3,1,1)$ \hspace{2cm}{$A=\begin{bmatrixcolor}[orange]
                          1 & -3   \\
                          2 & 1 \\
                          0 & 1       
                        \end{bmatrixcolor}$} \\

               \item [ii)]$v(-2,4)$ \\ $(-2-3(4),2(-2)+4,4)$ \\ $(-2-12,-4+4,4)$ \\\\ \textcolor{persianrose}{$\therefore imagen=(-14,0,4)$}\\\\
                         
                \item [iii)]   \includegraphics[width=7cm, height=3cm ]{ejercicio5(iii).png}
          
              \end{enumerate}





      \item \begin{enumerate}
              \item [i)]$T(\vec{e_{1}}) = T  (1,0)=(0,1)$ \\
              $T(\vec{e_{2}})=T(0,1) = (1,0)$ \hspace{3cm}{$B=\begin{bmatrixcolor}[bulgarianrose]
               1 & -3   \\
               2 & 1 \\
               0 & 1       
             \end{bmatrixcolor}$} \\

              \item [ii)] \textcolor{persianrose}{$\therefore imagen=(4,3)$}\\\\
              
              \item [iii)]
                \includegraphics[width=7cm, height=3cm ]{ejercicio5b(iii).png}
    

            \end{enumerate}        



      \end{enumerate}
 \newpage

\item Sea $T : R^3 \rightarrow R^3 $ una transformación lineal definida por $T(x, y, z)$ = $(x+y +z, 2z -x, 2y -z)$.      
      \begin{enumerate}
        \item Determina si es invertible y encuentra su inversa.
      \end{enumerate}
      Determina $T(4, −5, 10)$ usando:
      \begin{enumerate}
        \item [b)] la matriz estándar
        \item [c)] la matriz relativa a
      \end{enumerate}

      $B = \{(2, 0, 1),(0, 2, 1),(1, 2, 1)\}$ y  $B^{'} = \{(1, 1, 1),(1, 1, 0),(0, 1, 1)\}.$



      {\color{amber} \rule{\linewidth}{0.5mm} }

      \begin{enumerate}
        \item [a)] $A=\begin{bmatrixcolor}[sangria]
          1 & 1 & 1  \\
          -1 & 0 & 2\\
          0 & 2 & -1     
        \end{bmatrixcolor}$} \\\\

        $ A= \left[
          \begin{array}{ccc:ccc}
          1 &1 &1 &  1 & 0 & 0\\ 
          -1 &0 &2 &  0 & 1 & 0\\
          0 &2 &-1 &  0 & 0 & 1\\ 
          \end{array} \right]$ \hspace{.2cm}{$\xRightarrow{\mathit{F_{1}+F_{2}}}}$}\hspace{.2cm}{$\left[
            \begin{array}{ccc:ccc}
            1 &1 &1 &  1 & 0 & 0\\ 
            0 &1 &3 &  1 & 1 & 0\\
            0 &2 &-1 &  0 & 0 & 1\\ 
            \end{array} \right] $}\hspace{.2cm}{$\xRightarrow{\mathit{-2F_{2}+F_{3}}}}$} \\\\

            $\left[
            \begin{array}{ccc:ccc}
            1 &1 &1 &  1 & 0 & 0\\ 
            0 &1 &3 &  1 & 1 & 0\\
            0 &0 &-7 &  -2 & -2 & 1\\ 
            \end{array} \right] $\hspace{.2cm}{$\xRightarrow{\mathit{-F_{2}+F_{1}}}}$}\hspace{.2cm}{$\left[
              \begin{array}{ccc:ccc}
              1 &0 &-2 &  0 & -1 & 0\\ 
              0 &1 &3 &  1 & 1 & 0\\
              0 &0 &-7 &  2 & -2 & 1\\ 
              \end{array} \right] $}\hspace{.2cm}{$\xRightarrow{\mathit{-\frac{1}{7}F_{3}}}}$}\hspace{.2cm}{$\left[
                \begin{array}{ccc:ccc}
                1 &0 &-2 &  0 & -1 & 0\\ \\
                0 &1 &3 &  1 & 1 & 0\\\\
                0 &0 &1 &  \frac{2}{7} & \frac{2}{7} & -\frac{1}{7}\\ 
                \end{array} \right] $}\\\\

                $\xRightarrow{\mathit{-3F_{3}+F_{2}}}}$\hspace{.2cm}{$\left[
                  \begin{array}{ccc:ccc}
                  1 &0 &-2 &  0 & -1 & 0\\ \\
                  0 &1 &0 &  \frac{1}{7} & \frac{1}{7} & \frac{3}{7}\\\\
                  0 &0 &1 &  \frac{2}{7} & \frac{2}{7} & -\frac{1}{7}\\ 
                  \end{array} \right] $}\hspace{.2cm}{ $\xRightarrow{\mathit{2F_{3}+F_{1}}}}}$\hspace{.2cm}{$\left[
                    \begin{array}{ccc:ccc}
                    1 &0 &0 &  \frac{4}{7} & -\frac{3}{7} & -\frac{2}{7}\\ \\
                    0 &1 &0 &  \frac{1}{7} & \frac{1}{7} & \frac{3}{7}\\\\
                    0 &0 &1 &  \frac{2}{7} & \frac{2}{7} & -\frac{1}{7}\\ 
                    \end{array} \right] $}\\\\

                    $A^{1}=\begin{bmatrixcolor}[sangria]
                      \frac{4}{7} & -\frac{3}{7} & -\frac{2}{7}  \\\\
                      \frac{1}{7} & \frac{1}{7} & \frac{3}{7}\\\\
                      \frac{2}{7} & \frac{2}{7} & -\frac{1}{7}     
                    \end{bmatrixcolor}$} \hspace{2cm}{$T(\vec{v})=A\vec{v}=$}\hspace{.2cm}{$\begin{bmatrixcolor}[sangria]
                      \frac{4}{7} & -\frac{3}{7} & -\frac{2}{7}  \\\\
                      \frac{1}{7} & \frac{1}{7} & \frac{3}{7}\\\\
                      \frac{2}{7} & \frac{2}{7} & -\frac{1}{7}     
                    \end{bmatrixcolor}$} \hspace{.1cm}{$\begin{bmatrixcolor}[sangria]
                      x \\
                      y\\
                      z     
                    \end{bmatrixcolor}=$}\hspace{.2cm}{$\begin{bmatrixcolor}[sangria]
                      \frac{4}{7}x & -\frac{3}{7}y & -\frac{2}{7}z  \\\\
                      \frac{1}{7}x & \frac{1}{7}y & \frac{3}{7}z\\\\
                      \frac{2}{7}x & \frac{2}{7}y & -\frac{1}{7}z     
                    \end{bmatrixcolor}$} \\\\

                    \textcolor{persianrose}{$\therefore T^{-1}(x,y,z)=(\frac{4}{7}x-\frac{3}{7}y-\frac{2}{7}z,\frac{1}{7}x+\frac{1}{7}y+\frac{3}{7}z,\frac{2}{7}x+\frac{2}{7}y-\frac{1}{7}z)$ }\\\\
                
\newpage
        \item [b)]  $T(\vec{e_{1}}) = T(1,0,0)=(1,0,0)$\\
                    $T(\vec{e_{2}}) = T(0,1,0)=(0,0,0)$\\
                    $T(\vec{e_{3}}) = T(0,0,1)=(0,0,-1)$ \\\\$A=\begin{bmatrixcolor}[sangria]
                      1 & 0 & 0  \\
                      0 & 0 & 0\\
                      0 & 0 & -1     
                    \end{bmatrixcolor}$ \hspace{1cm}{$T(4,-5,10)$}\hspace{.3cm}{$\begin{bmatrixcolor}[sangria]
                      1 & 0 & 0  \\
                      0 & 0 & 0\\
                      0 & 0 & -1     
                    \end{bmatrixcolor}$ }$\begin{bmatrixcolor}[sangria]
                      4  \\
                      -5 \\
                      -10     
                    \end{bmatrixcolor}$ \hspace{.2cm}{$=\begin{bmatrixcolor}[sangria]
                      4  \\
                      0 \\
                      -10     
                    \end{bmatrixcolor}$}   \\\\
                    
                    \textcolor{persianrose}{$\therefore T(4,-5,10)=(4,0,-10)$}\\

                    \item [c)]  $T(\vec{v_{1}})T(2,0,1)=(3,0,-1)=3(1,1,1)+0(1,1,0)-1(0,1,1)$\\
                                $T(\vec{v_{2}})T(0,2,1)=(3,2,3)=3(1,1,1)+(1,1,0)+(0,1,1)$\\
                                $T(\vec{v_{3}})T(1,2,1)=(4,1,3)=4(1,1,1)+(1,1,0)+(0,1,1)$\\

                                $A=\begin{bmatrixcolor}[guppiegreen]
                                  9 & 0 & -2  \\
                                  9 & 8 & 6\\
                                  12 & 2 & 6     
                                \end{bmatrixcolor}$ } \hspace{1cm}{$T(4,-5,10)$ $=\begin{bmatrixcolor}[guppiegreen]
                                  9 & 0 & -2  \\
                                  9 & 8 & 6\\
                                  12 & 2 & 6     
                                \end{bmatrixcolor}$}$\begin{bmatrixcolor}[guppiegreen]
                                  4  \\
                                  -5 \\
                                  10     
                                \end{bmatrixcolor}$\hspace{.2cm}{\hspace{.2cm}{$=\begin{bmatrixcolor}[guppiegreen]
                                  16  \\
                                  56 \\
                                  48    
                                \end{bmatrixcolor}$} } \\\\
                                
                                \textcolor{persianrose}{$\therefore T(4,-5,10)=(16,56,48)$}

      \end{enumerate}

\end{enumerate}


\end{document}