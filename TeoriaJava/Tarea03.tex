\documentclass[a4paper,10pt]{article} 
\usepackage[top=2cm,bottom=2cm,left=2cm,rigth=2cm,heightrounded]{geometry}
\usepackage[utf8]{inputenc}
\usepackage{graphicx}
\usepackage{multirow} 
\usepackage[spanish]{babel}
\usepackage[usenames]{color}
\usepackage{dsfont}
\usepackage{amssymb}
\usepackage{amsmath}
\usepackage{bbding}  
\usepackage[dvipsnames]{xcolor}
\usepackage{csquotes}
\usepackage[export]{adjustbox}
\usepackage[all]{nowidow} 
\usepackage{csquotes} 
\everymath{\displaystyle}
\usepackage{setspace}
\usepackage[yyyymmdd]{datetime} 
\renewcommand{\dateseparator}{-} 
\usepackage{fancyhdr}
\usepackage[inline]{enumitem}
\usepackage{arydshln}
\usepackage[utf8]{inputenc}
\usepackage[table]{xcolor}
\usepackage{listings} %Codigo de cualquier lenguaje.
\setlength{\arrayulewidth}{1mm}
\setlength{\tabcolsep}{18pt}
\renewcommand{\arraystretch}{1.5}
\newcolumntype{s}{>{\columncolor[HTML]{AAACED}} p{3cm}}
\arrayulecolor[rgb]{1.0, 0.44, 0.37}


\makeatletter
\newcommand{\xRightarrow}[2][]{\ext@arrow 0359\Rightarrowfill@{#1}{#2}}
\makeatother
\pagecolor{honeydew} 
\color{white}
\definecolor{ballblue}{rgb}{0.13, 0.67, 0.8}
\definecolor{bittersweet}{rgb}{1.0, 0.44, 0.37}
\pagestyle{fancy} 
\fancyhead{}\renewcommand{\headrulewidth}{0pt} 
\fancyfoot[C]{} 
\fancyfoot[R]{\thepage} 
\newcommand{\note}[1]{\marginpar{\scriptsize \textcolor{red}{#1}}} 
\begin{document}
\fancyhead[C]{}
\begin{minipage}{0.295\textwidth} 
\raggedright
Introducción a Ciencias de la Computación 2021-1\\    
\footnotesize 
\colorbox[rgb]{0.13, 0.67, 0.8}{\textcolor{black}{Cruz González Irvin Javier}}
\textcolor[rgb]{1.0, 0.44, 0.37}{\medskip\hrule}
\end{minipage}
\begin{minipage}{0.4\textwidth} 
\centering 
\large 
\textbf{Tarea 3}\\ 
\normalsize 
Diseño Orientado a Objetos(implementación)\\
\end{minipage}
\begin{minipage}{0.295\textwidth} 
\raggedleft
\today\\ 
\footnotesize
N$^{\circ}$DE CUENTA: 31716198-2
 \textcolor[rgb]{1.0, 0.44, 0.37}{\medskip\hrule}
\end{minipage}



\begin{enumerate}
    \item Problema 1.
    \begin{enumerate}
        \item Escribe los objetos y los métodos.\\
        
        \begin{tabular}{|l|l|}
            \hline
            Objetos & Métodos\\
            \hline
            alumno & calificarTarea\\
            \hline
            profesor &  elaborarTarea\\
            \hline
                     & entregarTarea\\
            \hline
                    & recibirTarea \\
            \hline
                    & getAciertos()/setAciertos() \\ 
            \hline                     
            \end{tabular}
        
        \item Elabora las tarjetas de responsabilidad para el profesor\\
                
        \begin{tabular}{ |l|l| }
            \hline
            \multicolumn{2}{|c|}{Clase.Profesor} \\
            \hline
             & \\
            P &  CONSTRUCTOR TAREA.Inicializa los valores(se crea la tarea) \\
            Ú & recibirTarea-recibe la tarea del alumno \\
            B & calificarTarea-califica la tarea del alumno apartir de sus aciertos y me regresa su calificación \\
            L  & getAciertos() - muestra los aciertos de la tarea del alumno \\
            I  & setAciertos() - modifica los aciertos de la tarea del alumno \\
            \hline
            C &  ejercicios=5 \hspace{.5cm}{Almacena los ejercicios propuestos por el profesor}\\
            O &   aciertos \hspace{1.1cm}{Almacena los aciertos obtenidos por el alumno}            \\  
             &             \\
            \hline
          \end{tabular}

    \end{enumerate}



    \item Problema 2.
    
    \begin{enumerate}
        \item Escribe los objetos y los métodos\\
         
        \begin{tabular}{|l|l|}
            \hline
            Objetos & \hspace{3.5cm}{Métodos(Descripción)}\\
            \hline
            Cena & cenar()\hspace{2cm}{El comensal accede a la comida}\\
            \hline
            Comensal &  elegirComida()\hspace{.8cm}{El comensal elige la comida de cada tiempo}\\
            \hline
            Cocinero  & elegirTiempo()\hspace{.8cm}{El comensal elige en que tiempo desea comenzar a cenar}\\
            \hline 
                      & Cena()\hspace{2cm}{Constructor.Inicializamos valores }\\      
            \hline          
                      & Comensal()\hspace{1.3cm}{Constructor.Inicializamos valores}\\
             \hline                
            \end{tabular}\\
\newpage
      \item Elabora las tarjetas de responsabilidad para el comensal.\\
      
      \begin{tabular}{ |l|l| }
        \hline
        \multicolumn{2}{|c|}{Clase.Comensal} \\
        \hline
         & \\
         &  CONSTRUCTOR Comensal.Inicializa los valores(creamos al comensal) \\
         & cenar\hspace{3.3cm}{El comensal accede a la comida} \\
        P & elegirComida \hspace{2cm}{El comensal elige la comida de cada tiempo}\\
        U  & elegirTiempo \hspace{2cm}{El comensal elige en que tiempo desea comenzar a cenar}\\
        B  &  \\
        \hline
        L &  nombre \hspace{3.6cm}{Almacena el nombre del comensal}\\
        I &   primerTIempo \hspace{2.5cm}{Almacena la comida del primer tiempo} \\  
        C &   segundoTIempo \hspace{2.3cm}{Almacena la comida del segundo tiempo} \\
        O &   tercerTiempo \hspace{2.7cm}{Almacena la comida del tercer tiempo} \\     
         &   guarnición \hspace{3.2cm}{Almacena la guarnicion del tercer tiempo}   \\   
        \hline
      \end{tabular}\\\\


      \item ¿Cuáles son los atributos de esta clase?
      \begin{lstlisting}

        String primerTiempo;
        String segundoTiempo;
        String tercerTiempo;
        String guarnicion;       
               \end{lstlisting}

       \item Escribe en Java el constructor de la clase cena.
       \begin{lstlisting}
        
      public Cena(String primerTiempo, String segundoTiempo, 
                    String tercerTiempo, String guarnicion) {

      this.primerTiempo = primerTiempo;
      this.segundoTiempo = segundoTiempo;
      this.tercerTiempo = tercerTiempo;
      this.guarnicion = guarnicion;

  }
       \end{lstlisting}
       

      \item Escribe el encabezado en Java de un método de acceso que nos diga si el tercer tiempo es filete o pescado
      a la diabla.

      \begin{lstlisting}
        public String getTercerTiempo() {
          return this.tercerTiempo;
      }

      public static void main(String[] args) {
        Cena cena1 = new Cena("ensala o pasta", 
        "sopa o crema de verduras", "pescado a la diabla o filete");

        System.out.println(cena1.getTercerTiempo());

    }
      \end{lstlisting}
\newpage
      \item Escribe el encabezado de un método mutante que elija si la cena va a ser con ensalada o pasta.
      
      \begin{lstlisting}

        public void setPrimerTiempo(String primerTiempo) {
        this.primerTiempo = primerTiempo;
    }

        cena1.setPrimerTiempo("pasta");
        System.out.println(cena1.getPrimerTiempo());

        
      \end{lstlisting}

      \item Escribe en Java el método de implementación al que le damos como parámetros el tipo del segundo tiempo
      y del tercer tiempo y los actualiza.

      \begin{lstlisting}
        public boolean elegir(String primerTiempoo, String segundoTiempoo) {
          primerTiempoo = "ensalada"
  
          if (comensal == primerTiempoo) {
              return true;
          } else {
              return false;
          }
  
      }
      \end{lstlisting}
      
      


    \end{enumerate}
    



\end{enumerate}



\end{document}